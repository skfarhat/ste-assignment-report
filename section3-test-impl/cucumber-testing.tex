\subsection{Cucumber testing}

\subsubsection{File structure}

Briefly, cucumber testing works by first defining a feature under test in a feature file. A feature may have several scenarios with each scenario defining the given input, the steps defining a state transition, and the expected outcome from these transitions. Keywords: Given, And, Then, When are used in the feature file. 
\par
To translate those into actual code, Java methods with annotations (@Given, @And, @Then, @When) and a regular expression the step definition are created. Each method implements translates the step definition into the equivalent executing code and makes the assertions and checks it needs to. 
\par
\paragraph{File structure}
\begin{itemize}
    \item \javainline{src/test/java/testing/accpetance/*.java}: java test implementation for step definitions 
    \item \javainline{src/test/resources/*.feature}: feature files defining features under test. 
\end{itemize}

Each of the feature files in resources has their own FeatureXStepDefs which defines their methods. Except for Feature7. As Feature7 is not actually a feature but more of a non-functional requirement, it consists of scenarios which span all previous features. 

\subsubsection{Splitting the code into many classes - and CommonStepDefs}
As the number of feature scenarios grew, so did the number of methods implementing each step in those scenarios. 
We moved from having a single StepDefs.java class that defined all steps to separate FeatureXStepDefs classes each of which defined the steps particular to a single feature that is under test.
These FeatureXStepDefs classes shared lots of code in common, with regards to the helper methods they would need to call and the member variables they relied on to validate the app's state. 
\par
In light of this, a single java class (CommonStepDefs) housing all common code needed to be created and linked to the other FeatureXStepDefs classes. The class contains the following: 

\begin{itemize}
    \item \javainline{applicationHasStarted}
    \item \javainline{userEnters}
    \item Many other helper methods that are used by each Feature class such as 
    \begin{itemize}
        \item \javainline{resetOutStream}, 
        \item \javainline{closeApp}, 
        \item \javainline{waitForMainToBeReady}.
    \end{itemize}
\end{itemize}


\par
\paragraph{Linking Common class with Feature classes\\}
Inheritance was considered as a means of linking Common to Feature classes, however, cucumber does not support classes defining stepdefs to be inherited from as this creates multiple instances of the stepdef method making it impossible for Cucumber to choose which one to call during scenario execution. 
\par 
Next, we considered composing the Common class in each Feature class, and this was made possible through a cucumber dependency titled PicoContainer. With this library, cucumber is able to create a CommonStepDefs object for every Feature class and injecting it into the Feature constructor so long as the constructor was implemented to take in a CommonStepDefs object. Figure \ref{code:cucumber-piccontainer-di} is an extract of all that is required for this to be achieved.

\begin{figure}[H]
\begin{javacode}
/** injected through constructor by Cucumber PicoContainer */
private CommonStepDefs common;

/** constructor */ 
public Feature1StepDefs(CommonStepDefs common) {
    this.common = common;
}
\end{javacode}
\caption{Dependency injection - CommonStepDefs in Feature1StepDefs}    
\label{code:cucumber-piccontainer-di}
\end{figure}
% ---------------------------------------------------------------------------
 
% Cucumber test suite structure 
% The structure of the code in the Cucumber suite. 
% Refactoring of our Cucumber test suite. 
% there's a better way to test when adding customer and that is to get the Customer object back rather on relyin on the output of the command line to get the object back 


\subsubsection{Implementation details}

% Briefly go through what is needed to get this setup. 

% scenarios use applicationHasStarted
Many of our Scenarios (if not all) depend on applicationHasStarted(). This step starts the "Main" application. By design "Main" waits for user input to proceed and will block the execution thread until exits. To stop this, we make applicationHasStarted start "Main.run()" in a separate thread. 

% timing challenges in the tests in that we have to wait for the main thread and make sure it is accepting of user input before we inject something.  
% we added a waitingForCommand boolean field (of course, it's volatile too), 
% we use this to monitor if the Main thread is accepting of input
This introduces some timing challenges. Subsequent steps in our framework need to work with Main by injecting user input and monitoring the output, but the steps need to know if "Main" is currently accepting of user input and if it has printed the output it requires. 
% The solution to our user input / timing problem
One solution to this is to add a boolean field member `waitingForCommand` in "Main" which indicates that the app is currently waiting for the next command to execute. The framework can then know for certain if it can safely inject new input to System.in and whether it can assume that Main has printed all it needed to already. This latter is ensured by the fact that Main will not have anything more to print() when it is blocked waiting for a new command from user input. 

\subsubsection{Testing calculated tax values}

Time was taken to decide how to proceed with verifying correctness of tax values. JUnit offers a \linebreak @ParametrizedTest annotation that allows the calling of a test function multiple times with different parameters, each time checking the correctness of the output given the input parameters. This is not only useful but needed, it is far too costly time and effort wise to write a new tests for each combination of tax code and salary. (Provide calculation estimate here on how many there are approximately).
If the tests were to be spelled out individually (one test for each taxcode-salary combination), then it would require even more testing time if a new taxcode 'letter', or a new salary bracket were to be added in the future. (e.g. adding a new taxcode letter for people with 3 children exactly, would require xxxx tests more). 

For the above reason, we looked into parametrized tests using Cucumber and found this to be possible using the "Examples" keyword. 
An "Examples" section placed just below a scenario definition determines the combinations of parameters to be used when running this Scenario, for our tax verification purposes we use the table below: 
\begin{javacode}
firstName	| lastName	| taxCode 	| Salary 	| expectedTax
Tobias 		| Mann 		| TB1000 	| 10000 	| ....
Sophie		| Caseby	| S2000 	| 20000		| ....
\end{javacode}
This entails the test will be run with Tobias as firstName, Mann as lastName... for the first run of the scenario. Then for the second run "Sophie" will be used for the firstName and so on. 
This approach allows us to define fewer scenarios that can take in more combinations increasing our scenarios' readability all the while reducing the time needed to test the large number of combinations we need to. 

The changes needed for this to work on the Scenario's side of things is to use <firstname> and <lastname> in the steps in the feature file instead of hardcoding the values, and to add String parameters to the Java StepDefinitions.


\paragraph{ExamplesTableGenerator}
Explain how the tool is provided to help generate the Examples table that can be plugged into the .feature file. And how after some careful reading, it was determined that it's not a great solution (also it is not implemented in cucumber) to have the feature file load a separate examples file, because we want to have it all be contained. 
\par
While it may be argued that the approach taken here is overkill or not in the spirits of Behaviour Driven Testing, we would argue that it is our role to detect whatever defects we can and that the implications of defective code are very costly in our case, we have been asked to find the most defects using cucumber which is a BDD tool and we have taken the necessary approach that we believe achieves the objective. 

% Check for uniqueness. Cannot be done conclusively. 
\subsubsection{Testing for uniqueness requirement}
It's near impossible to fully confirm that the application is indeed returning unique identifier without incurring high computational costs. For example, the application may be coded to return a Random int ID in the range [1-100] for each customer. A test that creates 5 customers would still only have 10\% chance of hitting a uniqueness case: 
$1 - P(100,5) = 0.10$. This is known as the Birthday problem % https://en.wikipedia.org/wiki/Birthday_problem]. 
An application returning a random Integer. 


\subsubsection{We try to write a test for every defect, so that we can catch it in the future if it reoccurs.}
It is impossible to test against all possible scenarios. During the testing process, testers employ their rationale and experience when choosing what inputs to test for and cases to verify; this still leaves many cases untested.
\par
A common approach when a defect is detected one way or another, is to write a test that would catch the error if it were to reoccur in the future. Some of the defects found in the application were detected through code inspection, exploratory testing or unit testing. In all cases, we attempt to write a behavioural test that would reproduce the defect and catch it were to be introduced again in the future. 
