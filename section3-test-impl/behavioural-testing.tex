\subsubsection{Behavioural driven testing}

% -----------------------------------------------------------------------------
\paragraph{Introduction}
Cucumber provides a framework by which testers can define feature files that detail in common language the required features of the application as well as providing different scenarios while specifying the expected behaviour from the application in response to the scenarios. 
\par 
Feature files allow professional who are not familiar with programming the ability to both understand and write application features. 
These files are then linked to application specific step definitions (in our case written in Java), translating scenario steps to code. Java methods with annotations (@Given, @And, @Then, @When) and a regular expression allow the linking of feature file to Java classes.  
% -----------------------------------------------------------------------------
\paragraph{FeatureXStepDefs}
Step definitions defined within a class need to share data among them to accomplish their task; in Cucumber Java, this is most often done by defining member variables that can be accessed by methods within this same class.  
\par 
As the number of features and scenarios grew, so did the number of member variables created, making it impede code readability. In response to this, step definitions describing different features were separated into disparate classes (Feature1StepDefs, Feature2StepDefs..) each encapsulating their feature's own logic.  
\par 
% -----------------------------------------------------------------------------
\paragraph{CommonStepDefs} 
While useful for readability, the above separation of classes introduced new challenges: many of these features shared step definitions in common; and a step definition matching a regex could only be defined in one class and one class only.
\par 
A single class housing these common step definitions was hence required, such a class would need to link to all other feature classes: inheritance and composition were considered.
\par
Inheritance was first attempted but rejected by cucumber as it would cause step definition methods to be defined for each inheriting instance causing confusion for the cucumber on which method to invoke. 
\par
Composition on the other hand was possible, and aided by another cucumber dependency titled PicoContainer.  The library allows cucumber to create the object to be composed (CommonStepDefs in our case) and inject it into each of the Feature classes that requires it. From a code point of view, the only requirement on behalf of the Feature class would be to define a constructor taking in the object to be injected. Figure \ref{code:cucumber-piccontainer-di} is an extract of all that is required for this to be achieved. 
\begin{figure}[H]
\begin{javacode}
/** injected through constructor by Cucumber PicoContainer */
private CommonStepDefs common;

/** constructor */ 
public Feature1StepDefs(CommonStepDefs common) {
    this.common = common;
}
\end{javacode}
\caption{Dependency injection - CommonStepDefs in Feature1StepDefs}    
\label{code:cucumber-piccontainer-di}
\end{figure}
Our CommonStepDefs included the definition of step definitions that were commonly used by most scenarios as well as helper methods that were integral to the Feature class' functioning. 
\begin{itemize}
    \item \javainline{applicationHasStarted}
    \item \javainline{userEnters}
    \item \javainline{resetOutStream}, 
    \item \javainline{closeApp}, 
    \item \javainline{waitForMainToBeReady}.
    \item \javainline{theApplicationShouldNotCrash}
    \item \javainline{listOfCustomerDetails}
\end{itemize}

\paragraph{Scenario Outline}
Given the nature of the software under test and the range of possible combinations of inputs (tax-code, salary, base-amount) that are applicable, it was deemed infeasible time-effort wise to create a scenario for each combination. 
However as the input constraints are understood, long lists could be generated using another program. Hence we developed the Generator class along with ExampleTablesGenerator, which together can provide longs lists of input combinations using some constraint (invalid tax codes, valid taxcodes, taxcodes with negative salaries). 
\par 
The generated tables would then be plugged in to the "Examples" section in "Scenario Outline" allowing for a single Scenario definition to be executed by all the different examples. This feature allows for high code coverage using minimal additional effort. Our 8\_TaxRulesExtensive.feature contains +1000 different examples that are all run for one scenario. 

\paragraph{Correct implementation}
In order to compare the given application's behaviour to what would be expected in the correct case, effort was expended on developing working versions of the different classes in question. 
\par 
It is well acknowledged that this may not always be possible when testing as some of the code bases under test can consist of thousands of files. However, in our situation it was deemed feasible and beneficial in that it served to confirm either defects in the code under test or even defects in the written tests. 
\par 
The working versions of the different classes consist of the class names prefixed with "Correct": CorrectTaxEngine, CorrectMain... 
\par 
There is no proof that the test implementation is in fact correct, and cases where both generate the same output may well be cases where both implementations are defective.
