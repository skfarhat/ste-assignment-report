% TODO:  use surefire report
\subsection{Unit testing}

\begin{itemize}
	\item we use display names to give more readable output for tests
	\item we use parametrized tests to allow for high flexibility in generating the testCases. --> Explain the alternative where we would have to write an explicit test case for every input type. And further explains, why running a loop with asserts() in there is not equivalent to our current approach. (Basically the answer to the question, why did we use Paramterized testing). 
	\item we acknowledge that some regex guru may come up with a regex that replaces the 3 we currently use right now. That's not a problem, we opted for readability anyway ;) 
	\item we've made some effort to make the tests generalised in that. If some of the requirements change (change the letter used for a symbol) or if more letters are added, then we wouldn't have to overhaul the whole Test class but merely make some select changes. Of course, some tests take @ValueSource which are hardcoded strings to test. In case of change in requirements, those especially should be carefully reviewed (one by one). 
\end{itemize}