\subsubsection{Exploratory Testing}

Exploratory testing can be understood as a less structured form of testing that involves black-box testing the application in real time and aiming to discover defects in the software. 
This approach is often orders of magnitude more productive than structured testing and can involve a lot more creativity on behalf of the tester leading to certain code paths being visited, that may not have been otherwise; it is hence more likely to find new defects than structured scripted tests are.
\par
In the earlier stages where testers are beginning to learn about the application itself before testing it, exploratory testing consists of merely using the application and trying to understand its behaviour in response to the different inputs. 
\par
Furthermore, exploratory testing is much better suited to detecting defects that are difficult to detect in regular testing, unless the tester was actively looking for them. 
As an example, our early exploratory testing allowed us to detect many inconsistencies in the formatting of the output in the application that we would not have been necessarily tested for otherwise.  

\begin{itemize}
    \item \textbf{D007}: missing whitespace from the printed Menu table
    \item \textbf{D010}: wrong output 'customer deleted' when updating customer
    \item \textbf{D006}: Missing 'y' command for customer salary update when printing menu
\end{itemize}