% ==============================================================================

% ==============================================================================
% HEADER										
% ==============================================================================
% ==============================================================================
\documentclass[11pt]{article}
\author{Sami Farhat}
\title{STE}

% ------------------------------------------------------------------------------
% packages
\usepackage{titlesec} 				% to change the section font-size
\usepackage[margin=0.8in]{geometry} 	% for Margins
\usepackage{geometry}
\usepackage{hyperref}				% for links
\usepackage{multicol}				% for embedding lists in multiple-columns
\usepackage{enumitem}				% customize lists
\usepackage{fontspec}				% customise font used
\usepackage{tabularx}
\usepackage{soul}
\usepackage{color} 					% colouring text
\usepackage{array}					% bold table column
\usepackage{booktabs}				% http://ctan.org/pkg/booktabs
\usepackage{etoolbox}
\usepackage{float}
\usepackage[table]{colortbl}			% http://ctan.org/pkg/xcolor
\usepackage{tikz}					% for hierarchy structure
\usepackage[export]{adjustbox}
\usepackage{wrapfig}
\usepackage{graphicx}
\usepackage{longtable}
\usepackage{listings}
\usepackage{textcomp}
\usetikzlibrary{shapes,arrows}
\usepackage{subcaption}
\usepackage[perpage, bottom]{footmisc}		% restart footnote count on per page basis
\usepackage{calc}	% for math computations
\usepackage[nomessages]{fp} % http://ctan.org/pkg/fp
\usepackage{enumitem,amssymb}
\usepackage[fleqn]{amsmath}
\usepackage{mathtools}
\usepackage{unicode-math}
\usepackage{xcolor,listings}
\usepackage{textcomp}
\setmainfont{Helvetica}

% for line breaks in table
\newcolumntype{b}{X}
\newcolumntype{s}{>{\hsize=.5\hsize}l}

\newcommand{\specialcell}[2][l]{%
\begin{tabular}[#1]{@{}l@{}}#2\end{tabular}}

\newcommand{\tblheader}{\textbf}
\definecolor{listinggray}{gray}{0.9}
\definecolor{lbcolor}{rgb}{0.9,0.9,0.9}

% path to graphics 
\graphicspath{ {res/} }

% sections
\setcounter{section}{0}
% no numbering of sections 
%\setcounter{secnumdepth}{0}

% increase margins of tables (1 is default)
\def\arraystretch{1.5}

% remove paragraph indent
\setlength{\parindent}{0pt} 

\newcommand{\tblitembegin} {
\vspace{-\topsep}
\begin{itemize}[noitemsep, topsep=0pt,itemsep=-1ex,partopsep=1ex, parsep=1ex,leftmargin=*,]
}
\newcommand{\tblitemend} {
\end{itemize}
\vspace{-\topsep}
}

\lstset{
language=Java,
upquote=true, 
%breakatwhitespace=false,         % sets if automatic breaks should only happen at whitespace
breaklines=true,                 % sets automatic line breaking
showspaces=false,
showstringspaces=false,
basicstyle=\ttfamily,
%numbers=left,
%numberstyle=\tiny,
frame=single,
%commentstyle=\color{gray}
keywordstyle=\color[rgb]{0,0,1},
commentstyle=\color[rgb]{0.133,0.545,0.133},
stringstyle=\color[rgb]{0.627,0.126,0.941}
}

% ==============================================================================
% ==============================================================================
%  CONTENT									
% ==============================================================================
% ==============================================================================

\begin{document}

% ============================================================================== 
% TITLE PAGE
% ============================================================================== 

\clearpage
%\maketitle
\thispagestyle{empty} % remove page number
\pagebreak 
\begin{titlepage}
\begin{center}
\vspace*{1cm}
{\Huge Software Testing} \\
\vspace{0.5cm}
{\LARGE STE\\}
\vspace{0.5cm}
\vspace{3.5cm}
{\LARGE Sami Farhat\\}
\vspace{0.1cm}
{\Large \# 1065452\\}
\vspace{0.3cm}
\vfill
\vspace{0.8cm}
Software Engineering Programme\\        
Computer Science\\
\vspace{0.5cm}
University of Oxford\\
United Kingdom\\
\vspace{1.0cm}         

\includegraphics[width=0.15\textwidth]{oxford-logo.png}
\end{center}
\end{titlepage}


% TABLE OF CONTENTS
% ============================================================================== 

\clearpage
\tableofcontents 
\thispagestyle{empty} % remove page number
\pagebreak

% ============================================================================== 
% CONTENT
% ============================================================================== 

\pagenumbering{arabic} % start numbering  



% TODO 
% ----
%
% Place the requirements in a table and number them to allow for clear referencing. 


\section{Review: System Requirements}
% Talk about our impression from reading the system requirements and all the vaguenesses found in there. 

\section{Code Review}

\subsection{Code Read}
% Detail what notes were taken when reading the code and what was noticed from afar
% We should say that reading the code is not great in that it usually focuses our testing on some things and push us not to test, things that 
% we expect to be correct (given we know the implementation). As such, it is always better to try and make no assumptions about the contents 
% of the code when writing the tests. 

% DEFECT: There's somewhat of an assumption being made that a users salary always fits in Integer. What if they earned more than 2^31 - 1 ? 

\subsection{Code Refactoring}

% md5_refactor_check
% ------------------
% Talk about how we want to prevent our code refactoring from actually fixing hidden bugs. Of course we cannot practically be sure that it does not, but we have done is write a script that compares the stdout from certain sequences of commands from both the original code and the modified-refactored code. The script generates the md5 hash from each of the outputs. They should match. If they don't this points to us having modified the current output. This script has been added as a post-commit hook, which means that on commit made a Jenkins build runs.  


\subsubsection{Changes in TaxEngine.java}

% [1] create the exception type TaxEngineException
[1] First off, we are displeased with the error reporting of this class, it does not make testing for error conditions any easier. We create an "exceptions" package and add TaxEngineException which extends RuntimeException in this package. We will expect this exception object to be thrown in TaxEngine methods where errors occur. 

% why choose RuntimeException
We chose RuntimeException as it then does not require all methods to add the "throws TaxEngineException" which will be added to a lot of sections of the code causing many other changes. We opted for the most gains for the least upsets to the code here. 

% [2] split out the different branches in taxAmount() to smaller more testable methods
[2] The contents of every block similar to `if (strTaxCode.indexOf(Constants.MARRIED_CODE) >0 ) {` were moved to private methods, making for 8 in total (see below). As these methods are private, they cannot be tested without the usage of reflection or other Java dark arts. As such, we further create 3 public methods (see below).

In taxAmount(String, int), the changes involve replacing the contents of the different if blocks with the right method calls to taxBasedOnMaritalStatus(), taxBAsedOnChildren() and taxBasedOnEducationalStatus() each time passing the proper enum. 

% Just trying to explain that separating the big method into smaller ones that are individually testable is better, as it allows us to more quickly/easily locate bugs when the different tests fail. 
From a testing point of view, we still want to write tests on taxAmount with the different combinations of TaxCodes and input parameters, checking each time if the return is as expected. Doing  

But in addition to the above, we also write different tests for each of the [taxBasedOn...] method. With simpler methods in TaxEngine, and tests targeting those simple methods, we can more quickly identify the source of bugs when tests fail. The alternative 

Basically if there was something wrong with how we compute salaries for Married people, then I'd expect this to manifest in the test for taxAmount() and
in taxBasedOnMarriage(), these two methods failing together should indicate to the tester or developer checking the tests that it is likely that the logic
in Marriage taxing is causing that in taxAmount to fail. Having both higher level and lower level tests is useful in that way. 

% -----------------------------
% Methods added: 
% -----------------------------
* taxBasedOnMaritalStatus
* taxBasedOnChildren
* taxBasedOnEducationStatus  
% -----------------------------
* taxMarriedCode
* taxSingleCode
* taxDivorcedCode
* taxOneChildCode
* taxTwoChildrenCode
* taxMultipleChildrenCode
* taxFullTimeStudentUnder24
* taxFullTimeStudentOver24
% -----------------------------

\subsection{Changes in Main}

* Allow for the selection of PrintStream and UserInputStream to Main. If not set then System.in and System.out are used. 
We could have also just used System.setOut() and System.setIn(), but our approach should be even more flexible as it doesn't affect our usage of System.out in the testing framework. 

* Make methods non-static. The reason for them being static is unclear at present; it could be that a behaviour similar to the one provided by a Singleton pattern design is the sought one. We decide - aginast our natural tendency - not to change the static types of the methods since we can still work with them being static. 

* Allow for Main's start/stop execution to be controlled via methods. Currently, to stop main, the user must press 'q' which then leads to System.exit(). If running a test suite, this will cause the whole test suite to exit. We replace System.exit() with logic that causes the method to return. 
 -	In a normal running App,   when the method returns, the app will reach the end of Main and exit naturally (and gracefully), when run under a test suite, the framework will continue to run after the method has returned which is what we wanted in the first place. 
  - Add method `stop()` and member field `running` which when set to false causes the User Input loop to exit, leading to stop of the program.

\section{Test implementation}


% TODO:  use surefire report
\subsection{Unit testing}

* we use display names to give more readable output for tests
* we use parametrized tests to allow for high flexibility in generating the testCases. --> Explain the alternative where we would have to write an explicit test case for every input type. And further explains, why running a loop with asserts() in there is not equivalent to our current approach. (Basically the answer to the question, why did we use Paramterized testing). 
* we acknowledge that some regex guru may come up with a regex that replaces the 3 we currently use right now. That's not a problem, we opted for readability anyway ;) 
* we've made some effort to make the tests generalised in that. If some of the requirements change (change the letter used for a symbol) or if more letters are added, then we wouldn't have to overhaul the whole Test class but merely make some select changes. Of course, some tests take @ValueSource which are hardcoded strings to test. In case of change in requirements, those especially should be carefully reviewed (one by one). 


\subsection{Cucumber testing}

Many of our Scenarios (if not all) depend on applicationHasStarted(). This step starts the "Main" application. By design "Main" waits for user input to proceed and will block the execution thread until exits. To stop this, we make applicationHasStarted start "Main.run()" in a separate thread. 

This introduces some timing challenges. Subsequent steps in our framework need to work with Main by injecting user input and monitoring the output, but the steps need to know if "Main" is currently accepting of user input and if it has printed the output it requires. 

One solution to this is to add a boolean field member `waitingForCommand` in "Main" which indicates that the app is currently waiting for the next command to execute. The framework can then know for certain if it can safely inject new input to System.in and whether it can assume that Main has printed all it needed to already. This latter is ensured by the fact that Main will not have anything more to print() when it is blocked waiting for a new command from user input. 

\subsection{List of defects}

% Requirements defects 
% Logic defects
% User Interface defects
% Code style defects


% ============================================================================== 
% APPENDIX
% ============================================================================== 

\appendix

\setcounter{section}{0}
\pagebreak 
\clearpage
\thispagestyle{empty} % remove page number
\vspace*{9cm}
\begin{center}
{\bf \LARGE Appendix}
\end{center}
\vfill
\pagebreak

% \section{More details}


% ==============================================================================
% END DOCUMENT
% ==============================================================================

\end{document}

% ==============================================================================
% CHECKS
% ==============================================================================

% [  ] Check spelling
% [  ] Check with "grammarly"

% ==============================================================================
% DRAFT
% ==============================================================================