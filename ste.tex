% =============================================================================
% =============================================================================
% HEADER										
% =============================================================================
% =============================================================================

\documentclass[11pt]{article}
\author{Sami Farhat}
\title{STE}

% -----------------------------------------------------------------------------
% Packages

\usepackage{titlesec} 				% to change the section font-size
\usepackage[margin=0.8in]{geometry} 	% for Margins
\usepackage{geometry}
\usepackage{hyperref}				% for links
\usepackage{multicol}				% for embedding lists in multiple-columns
\usepackage{enumitem}				% customize lists
\usepackage{fontspec}				% customise font used
\usepackage{tabularx}
\usepackage{soul}
\usepackage{color} 					% colouring text
\usepackage{array}					% bold table column
\usepackage{booktabs}				% http://ctan.org/pkg/booktabs
\usepackage{etoolbox}
\usepackage{float}
\usepackage[table]{colortbl}			% http://ctan.org/pkg/xcolor
\usepackage{tikz}					% for hierarchy structure
\usepackage[export]{adjustbox}
\usepackage{wrapfig}
\usepackage{graphicx}
\usepackage{listings}

% \usepackage{longtable}

% \usepackage{textcomp}
% \usetikzlibrary{shapes,arrows}
% \usepackage{subcaption}
% \usepackage[perpage, bottom]{footmisc}		% restart footnote count on per page basis
% \usepackage{calc}	% for math computations
% \usepackage[nomessages]{fp} % http://ctan.org/pkg/fp
% \usepackage{enumitem,amssymb}
% \usepackage[fleqn]{amsmath}
% \usepackage{mathtools}
% \usepackage{unicode-math}
% \usepackage{xcolor,listings}
% \usepackage{textcomp}
\setmainfont{Helvetica}


% Items
% -----------------------

% Graphics
% ---------------------
\graphicspath{ {res/} }

% Sections
% ---------------------
\setcounter{section}{0}
% no numbering of sections 
%\setcounter{secnumdepth}{0}


\newcommand\subexercise{\@startsection{subsection}{2}{\z@}
{3ex\@plus 0ex \@minus 0ex}%
{0ex\@plus 0ex}%
{\normalfont\normalsize\bfseries}
}
\newcommand{\qheader}[2]{{
  \subexercise[#1]{#1 \hspace{1pt} #2\\}
  \noindent\rule{\textwidth}{0.5pt}
  \par
}}

% Tables
% ---------------------
% increase margins of tables (1 is default)
\def\arraystretch{1.5}

\newcommand{\tblitembegin} {
\vspace{-\topsep}
\begin{itemize}[noitemsep, topsep=0pt,itemsep=-1ex,partopsep=1ex, parsep=1ex,leftmargin=*,]
}
\newcommand{\tblitemend} {
\end{itemize}
\vspace{-\topsep}
}
% for line breaks in table
\newcolumntype{b}{X}
\newcolumntype{s}{>{\hsize=.5\hsize}l}

\newcommand{\specialcell}[2][l]{%
\begin{tabular}[#1]{@{}l@{}}#2\end{tabular}}

\newcommand{\tblheader}{\textbf}
\definecolor{listinggray}{gray}{0.9}
\definecolor{lbcolor}{rgb}{0.9,0.9,0.9}

% Paragraphs
% --------
% remove paragraph indent
\setlength{\parindent}{0pt} 
\parskip=10pt

% Code 
% ----
\lstset{
language=Java,
upquote=true, 
%breakatwhitespace=false,         % sets if automatic breaks should only happen at whitespace
breaklines=true,                 % sets automatic line breaking
showspaces=false,
showstringspaces=false,
basicstyle=\ttfamily,
%numbers=left,
%numberstyle=\tiny,
frame=single,
%commentstyle=\color{gray}
keywordstyle=\color[rgb]{0,0,1},
commentstyle=\color[rgb]{0.133,0.545,0.133},
stringstyle=\color[rgb]{0.627,0.126,0.941}
}

% =============================================================================
% =============================================================================
%  CONTENT									
% =============================================================================
% =============================================================================

\begin{document}

% ============================================================================ 
% TITLE PAGE
% ============================================================================ 

\clearpage
%\maketitle
\thispagestyle{empty} % remove page number
\pagebreak 
\begin{titlepage}
\begin{center}
\vspace*{1cm}
{\Huge Software Testing} \\
\vspace{0.5cm}
{\LARGE STE\\}
\vspace{0.5cm}
\vspace{3.5cm}
{\LARGE Sami Farhat\\}
\vspace{0.1cm}
{\Large \# 1065452\\}
\vspace{0.3cm}
\vfill
\vspace{0.8cm}
Software Engineering Programme\\        
Computer Science\\
\vspace{0.5cm}
University of Oxford\\
United Kingdom\\
\vspace{1.0cm}         

\includegraphics[width=0.15\textwidth]{oxford-logo.png}
\end{center}
\end{titlepage}

% ============================================================================ 
% TABLE OF CONTENTS
% ============================================================================ 

\clearpage
\tableofcontents 
\thispagestyle{empty} % remove page number
\pagebreak

% ============================================================================ 
% CONTENT
% ============================================================================ 

\pagenumbering{arabic} % start numbering  



% TODO 
% ----
%
% Place the requirements in a table and number them to allow for clear referencing. 


% REQUIREMENTS 
\section{Review: System Requirements}

% what type of requirements document is this, are they informal? 
It is not evident at the outset what type of requirements document these requirements belong in. If they are meant to be informal, we would kindly ask for a formal requirements specification to guide our testing, as without one, there will be muchroom for vagueness and assumptions to be made. This could lead to testing being done on a false understanding of the requirements and this in turn would be costly both in terms of money and time.  

% vocabulary not defined before the requirements. 
In formal requirements documents, key terms such as "will", "should", "must", and "may" are defined prior to the listing of the requirements in order to more precisely depict the requirement priorities.
It is good that the present requirements limit the key terms used to a subset of well-known ones (will, may, should, must) are used. This reduces the risk of vagueness emanating from (english understanding). 
Client and customer are used interchangeably in the requirements, we assume they refer to the same entity, but it would be good to put that in writing. Again "Definitions" section before the requirements would be the place to have that (IEEE).

% requirements organisation (bullets, no numbering)
The requirements are bulletized, a numbering scheme would be much more useful as that would allow for specific referencing in future discussions, documents and code. 
We will take it on ourselves to label the requirements for our own workflow, it is suggested that any additions or alterations needed on the requirements be done on the table (reference to table here). 
The first bullet in the list is not an explicit requirement, but an introductory sentence. This is misleading to the reader. 

% Some lack of coherence which lends to an overall lack of professionalism in the requirements 
Seventh bullet "The system must include a simple ‘help’ [...]" does not terminate with a full stop while the rest do. 
When outlining marital status taxing, we can see "Calculation of tax for Married people", "Calculation of tax for Single people", but the word 'divorced' is not capitalised, typo or lack of respect for divorced?  

% non-functional requirements. 
The only non-functional requirement related to the application's robustness. Should we assume there is no interest in security, performance? Surely this isn't the case for a Tax Calculation application. 

% DEFECTS and VAGUENESS in requirements 
% --------------------------------------

% 1 
[ The application must enable the storage of client’s personal details including.  ] 

There is a lack of specificity on what storage means here. Does it refer to persistent storage? In which, case a form of database on disk would be required, or is it sufficient for it to be stored in memory during application runtime? The latter would not offer persistent for if the application or machine crashed.  

If persistent storage is to be used, it would be useful for the requirements to ouline the requirements on such database, are there non-function requirements such as performance metrics that the database should respect? 

% 2 
[ The software should issue each customer a numeric identifier which must be unique to that client and never be used by a different client (that is, if a client is deleted from the system then their ID should not be used again but has to ‘die’ with them). ] 

This requirement is relatively more detailed than some of the others, given it is accompanied by an example detailing a scenario and how the application should behave with respect to that scenario. 

While 'die' is quoted, it is still an aggressive term to use in a requirement. 'Expire' is a reasonable alternative term, but we won't pedantize on this as there are more serious defects and vaguenesses in the requirements.

% 3
[ The PTMA should enable customer details to be updated. ]

This requirement can be further expanded on, especially considering that it relates to significant chunks of the code. 
What details can a user update? Should it be allowed to update a customer's account number (ID)?
Is it required that there be a separate menu command for updating each of the following: first name, last name, tax code ? What about salary? Was this a desirable or required feature?    

% 4
[ The application should be able to print lists of customer information with users being able to select blocks of customers to be printed based on their unique identifiers. ]

This requirements weakness only becomes evident when the application is tested and it becomes apparent that the Unique Identifiers generated for the created customers are never presented to the the application user for them to take note of. 

This requirement can further do with specificity to what type of block can be selected in the list to print. 
Should it be possible to use a comma-separated list of blocks to print out? This approach is not uncommon in other computer applications (e.g. specifying the list of pages to print on modern OSes is possible with "2-10,15-35"). This requirement could do with more specificity to the pages format. 

% 5
[ The application will calculate the tax amount to be paid by each client and display this
amount along with the other customer details. The amount of the tax is based on their tax
code and their current salary according to the rules below. ]



% requiremrents lacking more explanation (e.g. 3)
no expanding on how the update process should be. It seems that we want to allow the flexibility to update either of the customer's fields directly with its own command, but this does not link back to an evident requirement from the bulletised list. 

% MISSING REQUIREMENTS ? 

% showing the ID of created clients 
Requirement 4 points to the need to allow users to see customers based on their IDs, but there is no reference to how the application-generated IDs are presented to the user. This was found to be an issue during exploratory testing, and was sourced back to both a defect in the implementation and absence in the requirements. 

% no requirement to delete clients. 
There is no requirement about the user's ability to delete customers from storage. This is tacitly mentioned as part of the second requirements relating to ID uniqueness. We believe it is then an oversight on the requirements writers. 

% Adjustments and assumptions we make on the requirements. 
% Requests for improvements on the requirements. 

% Talk about our impression from reading the system requirements and all the vaguenesses found in there. 

\section{Code Review}

\subsection{Code Read}
% Detail what notes were taken when reading the code and what was noticed from afar
% We should say that reading the code is not great in that it usually focuses our testing on some things and push us not to test, things that 
% we expect to be correct (given we know the implementation). As such, it is always better to try and make no assumptions about the contents 
% of the code when writing the tests. 

% DEFECT: There's somewhat of an assumption being made that a users salary always fits in Integer. What if they earned more than 2^31 - 1 ? 

\subsection{Code Refactoring}

% say that in our changes we aim not to impact the behaviour of existing applications that depend on the current implementation, so older apps still have to work. 

% md5_refactor_check
% ------------------
% Talk about how we want to prevent our code refactoring from actually fixing hidden bugs. Of course we cannot practically be sure that it does not, but we have done is write a script that compares the stdout from certain sequences of commands from both the original code and the modified-refactored code. The script generates the md5 hash from each of the outputs. They should match. If they don't this points to us having modified the current output. This script has been added as a post-commit hook, which means that on commit made a Jenkins build runs.  


\subsubsection{Changes in TaxEngine.java}

% [1] create the exception type TaxEngineException
[1] First off, we are displeased with the error reporting of this class, it does not make testing for error conditions any easier. We create an "exceptions" package and add TaxEngineException which extends RuntimeException in this package. We will expect this exception object to be thrown in TaxEngine methods where errors occur. 

% why choose RuntimeException
We chose RuntimeException as it then does not require all methods to add the "throws TaxEngineException" which will be added to a lot of sections of the code causing many other changes. We opted for the most gains for the least upsets to the code here. 

% [2] split out the different branches in taxAmount() to smaller more testable methods
% [2] The contents of every block similar to `if (strTaxCode.indexOf(Constants.MARRIED_CODE) >0 ) {` were moved to private methods, making for 8 in total (see below). As these methods are private, they cannot be tested without the usage of reflection or other Java dark arts. As such, we further create 3 public methods (see below).

In taxAmount(String, int), the changes involve replacing the contents of the different if blocks with the right method calls to taxBasedOnMaritalStatus(), taxBAsedOnChildren() and taxBasedOnEducationalStatus() each time passing the proper enum. 

% Just trying to explain that separating the big method into smaller ones that are individually testable is better, as it allows us to more quickly/easily locate bugs when the different tests fail. 
From a testing point of view, we still want to write tests on taxAmount with the different combinations of TaxCodes and input parameters, checking each time if the return is as expected. Doing  

But in addition to the above, we also write different tests for each of the [taxBasedOn...] method. With simpler methods in TaxEngine, and tests targeting those simple methods, we can more quickly identify the source of bugs when tests fail. The alternative 

Basically if there was something wrong with how we compute salaries for Married people, then I'd expect this to manifest in the test for taxAmount() and
in taxBasedOnMarriage(), these two methods failing together should indicate to the tester or developer checking the tests that it is likely that the logic
in Marriage taxing is causing that in taxAmount to fail. Having both higher level and lower level tests is useful in that way. 

% -----------------------------
% Methods added: 
% -----------------------------
* taxBasedOnMaritalStatus
* taxBasedOnChildren
* taxBasedOnEducationStatus  
% -----------------------------
* taxMarriedCode
* taxSingleCode
* taxDivorcedCode
* taxOneChildCode
* taxTwoChildrenCode
* taxMultipleChildrenCode
* taxFullTimeStudentUnder24
* taxFullTimeStudentOver24
% -----------------------------

\subsection{Changes in Main}

% this is not a change that we've made, just a note that we have redirected stdin and stdout of the application 
% in order to capture and analyse its output and to control its input. 
* Allow for the selection of PrintStream and UserInputStream to Main. If not set then System.in and System.out are used. 
We could have also just used System.setOut() and System.setIn(), but our approach should be even more flexible as it doesn't affect our usage of System.out in the testing framework. 

% * Make methods non-static. The reason for them being static is unclear at present; it could be that a behaviour similar to the one provided by a Singleton pattern design is the sought one. We decide - aginast our natural tendency - not to change the static types of the methods since we can still work with them being static. 

* Allow for Main's start/stop execution to be controlled via methods. Currently, to stop main, the user must press 'q' which then leads to System.exit(). If running a test suite, this will cause the whole test suite to exit. We replace System.exit() with logic that causes the method to return. 
 -	In a normal running App,   when the method returns, the app will reach the end of Main and exit naturally (and gracefully), when run under a test suite, the framework will continue to run after the method has returned which is what we wanted in the first place. 
  - Add method `stop()` and member field `running` which when set to false causes the User Input loop to exit, leading to stop of the program.

% print the ID of the created Customer otherwise there is no way to know how to list that specific customer. (see newCustomer, has the 'id' of the new customer printed.)

% add method getCustomer in Main to aid with testing. As we would prefer getting the customer object from main directly rather than going through the route of listing customers with that ID, and serializing a customer object from the printed fields to use in our tests. e.g. adding "Sami Farhat TB1000 10000", then the ID is output, the we would have to, use 'l' <id>, and get the output from the program and serialise a Customer object from the output. Also, a bug in the listing would cause all tests depending on getting a customer from the database to fail. To separate the tests better, we create a method in Main that returns a customer object directly from the data structure in AllCustomers given a CustomerID. 


% Change in AllCustomers.java

Pass PrintStream 'out' via DI injection too. If none provided, then System.out is used (default). This is needed since the AllCustomers class is responsible for printing lists of customers.  


\section{Test implementation}

\subsection{Exploratory Testing}

* we notice the table is not very well formatted, the last line's '|' is not properly aligned with the rest. This is very difficult to detect in testing, and one would have to be looking for it to find it, really. 

* it prints out "Customer deleted when doing an update, or something similar". 


% TODO:  use surefire report
\subsection{Unit testing}

* we use display names to give more readable output for tests
* we use parametrized tests to allow for high flexibility in generating the testCases. --> Explain the alternative where we would have to write an explicit test case for every input type. And further explains, why running a loop with asserts() in there is not equivalent to our current approach. (Basically the answer to the question, why did we use Paramterized testing). 
* we acknowledge that some regex guru may come up with a regex that replaces the 3 we currently use right now. That's not a problem, we opted for readability anyway ;) 
* we've made some effort to make the tests generalised in that. If some of the requirements change (change the letter used for a symbol) or if more letters are added, then we wouldn't have to overhaul the whole Test class but merely make some select changes. Of course, some tests take @ValueSource which are hardcoded strings to test. In case of change in requirements, those especially should be carefully reviewed (one by one). 


\subsection{Cucumber testing}

% not A LOT OF REFACTORING was done 
Explain that, we questioned how much refactoring we should be doing before starting the testing and opted on the conservative end of the spectrum deciding not to do too much and fix the existing code. Our role is to test it, and it may even be that we have been hired to highlight how many defects existed so that administration takes action on the developers (oops), in which case we must stick to our defined role and do testing the best we can given what we are given. 

Other changes and refactoring could have been done to Main, such as setting up an event communication loop that would allow the thread in which Main ran to synchronize with the cucumber thread that is doing the testing. We opted against making major changes. 

Method getCustomer was added to main, this was needed for testing. It acts as a proxy to allCustomers. An alternative approach could have been to return a getter to the allCustomers private field, or to simply make that field public. But as only one feature of allCustomers is needed for testing (at the time of writing at least), we went ahead an exposed the single method we care for from allCustomers. 

Many of our Scenarios (if not all) depend on applicationHasStarted(). This step starts the "Main" application. By design "Main" waits for user input to proceed and will block the execution thread until exits. To stop this, we make applicationHasStarted start "Main.run()" in a separate thread. 

This introduces some timing challenges. Subsequent steps in our framework need to work with Main by injecting user input and monitoring the output, but the steps need to know if "Main" is currently accepting of user input and if it has printed the output it requires. 

One solution to this is to add a boolean field member `waitingForCommand` in "Main" which indicates that the app is currently waiting for the next command to execute. The framework can then know for certain if it can safely inject new input to System.in and whether it can assume that Main has printed all it needed to already. This latter is ensured by the fact that Main will not have anything more to print() when it is blocked waiting for a new command from user input. 

% note on checking uniqueness
It's near impossible to fully confirm that the application is indeed returning unique identifier without incurring high computational costs. For example, the application may be coded to return a Random int ID in the range [1-100] for each customer. A test that creates 5 customers would still only have 10\% chance of hitting a uniqueness case: 
$1 - P(100,5) = 0.10$. This is known as the Birthday problem % https://en.wikipedia.org/wiki/Birthday_problem]. 
An application returning a random Integer. 

% The concurrency issues observed 
Whilst doing cucumber testing we noted, that the Cucumber suite ran tests in parallel and while this is desired to reduce the time it takes to run tests, it introduced multiple threading issues. With Main class being static, all threads effectively shared the same PrintStream  - as well. [ More to say ? ]

We then decided to refactor both our test suite, which at the time comprised of one class containing all Step Definitions, and the Main class. Main was moved from being static based, to instance based; all static modifiers were removed, a method run() was created. Main constructors were introduced: a default one that would use the standard (default) stdin and stdout, and another that can have those provided as a dependency injection. This greatly increased flexibility and our ability to test the different Main instances independently. 

% NOTE: refactoring Main to become non-static introduces an issue with the customer identifier index being static 

% Refactoring of our Cucumber test suite. 
We moved from having a single StepDefs.java class that contained all steps to creating a separated FeatureXStepDefs class for each feature undergoing acceptance testing. Those classes shared a lot in common in terms of the helper functions they would need to call, but also of the member variables they would use to test the app's behaviour. For this reason, we moved these common members and helper functions into their own class BaseStepDefs, which then became the inheritance parent of each FeatureXStepDefs. 

% challenge with ApplicationHasStarted 
As all features defined in .features rely on the Given "Application has started", we first thought it logical to move the corresponding step definition method into BaseStepDefs; alas, cucumber complained that a class defining a Step Definition cannot be inherited from, so this could not be done. Our workaround, was to defined applicationHasStarted() as a helper in BaseSteps and have a separate regex "application has started" for each feature:
% * in 1_ClientAdd.feature, there's  Application has started1"
% * in 2_UniqueID.feature, there's Application has started2"
... 
Each of those step definitions then calls the method applicationHasStarted() from BaseStepDefs, which avoids code redundancy. 

This approach may not be perfect, but (1) it does the job, (2) it avoids code duplication, (3) it is not very confusing. 

% there's a better way to test when adding customer and that is to get the Customer object back rather on relyin on the output of the command line to get the object back 
\subsection{Testing calculated tax values}

Time was taken to decide how to proceed with verifying correctness of tax values. JUnit offers @ParametrizedTest annotation that allows the calling of a test function multiple times with different parameters, each time checking the correctness of the output given the input parameters. This is not only useful but needed, it is far too costly time and effort wise to write a new tests for each combination of tax code and salary. (Provide calculation estimate here on how many there are approximately).
If the tests were to be spelled out individually (one test for each taxcode-salary combination), then it would require even more testing time if a new taxcode 'letter', or a new salary bracket were to be added in the future. (e.g. adding a new taxcode letter for people with 3 children exactly, would require xxxx tests more). 

For the above reason, we looked into parametrized tests using Cucumber and found this to be possible using the "Examples" keyword. 
An "Examples" section placed just below a scenario definition determines the combinations of parameters to be used when running this Scenario, for our tax verification purposes we use the table below: 

firstName	| lastName	| taxCode 	| Salary 	| expectedTax
Tobias 		| Mann 		| TB1000 	| 10000 	| ....
Sophie		| Caseby	| S2000 	| 20000		| ....

This entails the test will be run with Tobias as firstName, Mann as lastName... for the first run of the scenario. Then for the second run "Sophie" will be used for the firstName and so on. 
This approach allows us to define fewer scenarios that can take in more combinations increasing our scenarios' readability all the while reducing the time needed to test the large number of combinations we need to. 

The changes needed for this to work on the Scenario's side of things is to use <firstname> and <lastname> in the steps in the feature file instead of hardcoding the values, and to add String parameters to the Java StepDefinitions.

% ExamplesTableGenerator 
% 
Explain how the tool is provided to help generate the Examples table that can be plugged into the .feature file. And how after some careful reading, it was determined that it's not a great solution (also it is not implemented in cucumber) to have the feature file load a separate examples file, because we want to have it all be contained. 

While it may be argued that the approach taken here is overkill or not in the sprits of Behaviour Driven Testing, we would argue that it is our role to detect whatever defects we can and that the implications of defective code are very costly in our case, we have been asked to find the most defects using cucumber which is a BDD tool and we have taken the necessary approach that we believe achieves the objective. 


\subsection{List of defects}

% Requirements defects 
% Logic defects
% User Interface defects
% Code style defects


% ============================================================================ 
% APPENDIX
% ============================================================================ 

% =============================================================================
% APPENDIX
\appendix
\setcounter{section}{0}
\pagebreak 
\clearpage
\thispagestyle{empty} % remove page number
\vspace*{9cm}
\begin{center}
{\bf \LARGE Appendix}
\end{center}
\vfill
\pagebreak
% =============================================================================

% =============================================================================
% LABELLED REQUIREMENTS

\section{Labelled Requirements}
\label{app:labelled-requirements}
% =============================================================================
    \begin{table}[H]
    \small
    \centering
    \begin{tabularx}{\textwidth}{| c | X |}
    \hline %  -----------   Header  ---------------------------------
    \tblheader{Label} & \tblheader{Requirement} \\
    \hline %  -----------   Header  ---------------------------------
    \label{req:r1}
    R1 & The application must enable the storage of client’s personal details including: 
    \begin{itemize}[itemsep=\tableitemsep, leftmargin=\tableleftsep]
        \item First Name
        \item Last (family) name
        \item Current salary (as a whole number in GB pounds. No ‘pence’ required)
        \item Tax code 
    \end{itemize}
    \\
    \hline % ---------------------------------------------------------
    \label{req:r2}
    R2 &  The software should issue each customer a numeric identifier which must be unique to that client and never be used by a different client (that is, if a client is deleted from the system then their ID should not be used again but has to ‘die’ with them). \\
    \hline % ---------------------------------------------------------
    \label{req:r3}
    R3 & The PTMA should enable customer details to be updated. \\
    \hline % ---------------------------------------------------------
    \label{req:r4}
    R4 & The application should be able to print lists of customer information with users being able to select blocks of customers to be printed based on their unique identifiers. \\
    \hline % ---------------------------------------------------------
    \label{req:r5}
    R5 & The application will calculate the tax amount to be paid by each client and display this amount along with the other customer details. The amount of the tax is based on their tax code and their current salary according to the rules below. \\
    \hline % ---------------------------------------------------------
    \label{req:r6}
    R6 & The system must include a simple ‘help’ system that lists all commands \\
    \hline % ---------------------------------------------------------
    \label{req:r7}
    R7 & The system must not crash if the user enters something that they are not meant to. \\
    \hline % ---------------------------------------------------------
    & The tax rules are as follows:\\
    \hline % ---------------------------------------------------------
    \label{req:r8-1}
    R8.1 & Tax amount can be calculated from a person’s tax code and their current salary. \\
    \hline % ---------------------------------------------------------
    \label{req:r8-2}
    R8.2 & A tax code has a numerical part and an alphabetic part. For compatibility with legacy
    tax code policy, the numeric and alphabetic portion of the tax code may be presented in either order. Thus, for clarity, both of the following tax codes are valid:
    \begin{itemize}[itemsep=\tableitemsep, leftmargin=\tableleftsep]
    \item 1080MT
    \item SF980
\end{itemize}
    \\
    \hline % ---------------------------------------------------------
    \label{req:r8-3}
    R8.3 & The numeric part of the tax codes represents a ‘base’ tax amount that is then modified by the alphabetic portion of the tax code and depending on the person’s current salary. \\
    \hline %  -----------   Header  ---------------------------------
    \label{req:r8-4}
    R8.4 & The alphabetic portion of the code may consist of multiple letters drawn from the following set, with their corresponding meanings: 
    \begin{itemize}[itemsep=\tableitemsep, leftmargin=\tableleftsep]
    \item ‘M’ : Married
    \item ‘S’ : Single
    \item ‘D’ : Divorced
    \item ‘C’ : Has a single child
    \item ‘E’ : Has two children
    \item ‘F’ : Has multiple (more than two) children 
    \item ‘T’ : Full-time student below the age of 24 
    \item ‘U’ : Full time student 24 years and older
\end{itemize}
    \\
    \hline % ---------------------------------------------------------
    \end{tabularx}
    \end{table}

    % ----------------------------> 
    % Split the table to two pages 
    % ----------------------------> 

    \begin{table}[H]
    \small
    \centering
    \begin{tabularx}{\textwidth}{| c | X |}
    \hline %  -----------   Header  ---------------------------------
    \tblheader{Label} & \tblheader{Requirement} \\
    \hline % ---------------------------------------------------------
    \label{req:r8-5}
    R8.5 & Calculation of tax for Married people: one of the following four bands will apply:
    \begin{itemize}[itemsep=\tableitemsep, leftmargin=\tableleftsep]
        \item Married people with a current salary of less than £5000 will pay a tax rate of
        70\% of their base tax amount.
        \item Married people with a current salary less than £12,000 will pay a tax rate of
        90\% of their base tax amount.
        \item Married people with a current salary less than £27,000 will pay a tax rate of
        120\% of their base tax amount
        \item Married people with a current salary above than £27,000 will pay a tax rate
        of 130\% of their base tax amount
    \end{itemize}
    \\
    \hline % ---------------------------------------------------------
    \label{req:r8-6}
    R8.6 & Calculation of tax for Single people: one of the following four bands will apply:
    \begin{itemize}[itemsep=\tableitemsep, leftmargin=\tableleftsep]
        \item Single people with a current salary of less than £6500 will pay a tax rate of 75\% of their base tax amount.
        \item Single people with a current salary less than £11,000 will pay a tax rate of 95\% of their base tax amount.
        \item Single people with a current salary above £22,000 will pay a tax rate of 125\% of their base tax amount
        \item Single people with a current salary above than £22,000 will pay a tax rate of 132\% of their base tax amount
    \end{itemize}
    \\
    \hline % ---------------------------------------------------------
    \label{req:r8-7}
    R8.7 & Calculation of tax for divorced people: one of the following four bands will apply:
    \begin{itemize}[itemsep=\tableitemsep, leftmargin=\tableleftsep]
        \item Divorced people with a current salary of less than £7200 will pay a tax rate
        of 60\% of their base tax amount.
        \item Divorced people with a current salary less than £13,000 will pay a tax rate of
        80\% of their base tax amount.
        \item Divorced people with a current salary above £24,000 will pay a tax rate of
        95\% of their base tax amount
        \item Divorced people with a current salary above than £24,000 will pay a tax rate
        of 110\% of their base tax amount
    \end{itemize}
    \\
    \hline % ---------------------------------------------------------
    \label{req:r8-8}
    R8.8 & Calculation of tax for people with a single child. One of the following three bands will apply:
    \begin{itemize}[itemsep=\tableitemsep, leftmargin=\tableleftsep]
        \item People with a single child and a current salary of less than £8000 will pay a tax rate of 80\% of their base tax amount after taking into consideration any
        adjustments due to their marital status
        \item People with a single child and a current salary of less than £10400 will pay a
        tax rate of 85\% of their base tax amount after taking into consideration any
        adjustments due to their marital status
        \item People with a single child and a current salary greater than £10400 will pay a
        tax rate of 95\% of their base tax amount after taking into consideration any adjustments due to their marital status
    \end{itemize}
    \\
    \hline % ---------------------------------------------------------
\end{tabularx}
\end{table}
    % ----------------------------> 
    % Split the table to two pages 
    % ----------------------------> 

\begin{table}[H]
\small
\centering
\begin{tabularx}{\textwidth}{| c | X |}
    \hline % ---------------------------------------------------------
    \label{req:r8-9}
    R8.9 & Calculation of tax for people with two children. One of the following three bands will apply:
    \begin{itemize}[itemsep=\tableitemsep, leftmargin=\tableleftsep]
        \item People with a two children and a current salary of less than £7400 will pay a tax rate of 90\% of their base tax amount after taking into consideration any adjustments due to their marital status
        \item People with a two children and a current salary of less than £9900 will pay a tax rate of 95\% of their base tax amount after taking into consideration any adjustments due to their marital status
        \item People with a two children and a current salary greater than £9900 will pay a tax rate of 101\% of their base tax amount after taking into consideration any adjustments due to their marital status
    \end{itemize}
    \\
    \hline % ---------------------------------------------------------
    \label{req:r8-10}
    R8.10 & Calculation of tax for people with more than two children. One of the following three bands will apply:
    \begin{itemize}[itemsep=\tableitemsep, leftmargin=\tableleftsep]
        \item People with more than two children and a current salary of less than £7000 will pay a tax rate of 120\% of their base tax amount after taking into consideration any adjustments due to their marital status
        \item People with more than two children and a current salary of less than £9000 will pay a tax rate of 125\% of their base tax amount after taking into consideration any adjustments due to their marital status
        \item People with more two children and a current salary greater than £9000 will pay a tax rate of 130\% of their base tax amount after taking into consideration any adjustments due to their marital status
    \end{itemize}
    \\
    \hline % ---------------------------------------------------------
    \label{req:r8-11}
    R8.11 & Calculation for people who are in full time education who are below that age of 24 : One of the following three bands will apply (adjustments to be made after taking into account marital status and number of children):
    \begin{itemize}[itemsep=\tableitemsep, leftmargin=\tableleftsep]
        \item People in full-time education below the age of 24 and have a current salary less than £2000 will pay a tax rate of 107\%
        \item People in full-time education below the age of 24 and have a current salary less than £3000 will pay a tax rate of 113\%
        \item People in full-time education below the age of 24 and have a current salary more than £3000 will pay a tax rate of 123\%
    \end{itemize}
    \\
    \hline % ---------------------------------------------------------
    \label{req:r8-12}
    R8.12 & Calculation for people who are in full time education who are 24 years of age and above : One of the following three bands will apply (adjustments to be made after taking into account marital status and number of children):
    \begin{itemize}[itemsep=\tableitemsep, leftmargin=\tableleftsep]
    \item People in full-time education, who are 24 and above and have a current salary less than £2000 will pay a tax rate of 109\%
    \item People in full-time education, who are 24 and above and have a current salary less than £3000 will pay a tax rate of 115\%
    \item People in full-time education , who are 24 and above and have a current salary more than £3000 will pay a tax rate of 125\%
    \end{itemize}
    \\
    \hline % ---------------------------------------------------------
\end{tabularx}
\caption{Requirements table}
\end{table}


% =============================================================================
% CODE CRITIQUE

% =============================================================================
\pagebreak
\section{Code critique}
\label{app:code-critique}

Due to timing constraints, this section is not comprehensive, but still constructive critique of the current code implementation. Issues pointed here do not reference functional defects of the code but rather flaws in code style, design and good coding practice. 

\subsection{Documentation}
There is a very clear lack of documentation across the source code. Methods do not have javadoc, most method implementations do no supply them either excepting for a few inline comments. 

\subsection{Code Style}

\paragraph{General}
\begin{itemize}
    \item Some methods use spaces between open bracket and first param and closing bracket and last parameter. Other methods do not obey this. 
    \item Some methods use canonical camel case while others break this (newcustomer vs. addCustomer)

    Unusual usages: 
    \begin{itemize}
        \item \javainline{String name = new String()} is not often seen as it is sufficient to use `String name = "";' which is more efficient. 
        \item \javainline{int i = 0; then using for(i=0; i < n; i++)}. As i is not used outside of the loop, it does not need to be defined outside. 
    \end{itemize}

    \item Developers should learn about non-capturing groups in regex and improve their regular expression skills (check CorrectTaxEngine for a better way to match the tax code input as defined).

    \item String handling can be improved by using StringBuffer (improvement)
    
\end{itemize}    

% -------------------
\paragraph{AccountNumbers}

\begin{itemize}
    \item Is not a well implemented Singleton pattern because the constructor is made protected and not private, meaning other classes in the same package can still instantiate it.
\end{itemize}

% -------------------
\paragraph{AllCustomers}
\begin{itemize}
    \item AllCustomers is a badly named class, CustomerManagement is a better alternative. 

    \item method getCustomer() returns a Customer object that has an "INVALID\_ID" when no such customer is present. This is unusual in that most common implementations would either throw an exception or return null. This implementation would be fine and accepted were it to be more documented. 

    \item updateCustomerXXXXX: all of these methods start off finding the customer, we should create a private method that returns the customer for a given ID. Also, discussing the current implementation of finding that customer; the below: 
\end{itemize}

\begin{javacode}
int i = 0;
int intCustomerIndex = 0;
boolean blFound = false;
for (i=0; i< intCurrentCustomerIndex; i++) {
    if (customers[i].getAccountNum() == intCustomerID) {
        blFound = true;
        intCustomerIndex = i;
        break;
    }
}
if (blFound == true) {
    customers[intCustomerIndex].setfirstName( strFirstName);
}
\end{javacode}
can be written as below: 
\begin{javacode}
int intCustomerIndex = -1;
for (int i=0; i< intCurrentCustomerIndex; i++) {
    if (customers[i].getAccountNum() == intCustomerID) {
        intCustomerIndex = i;
        break;
    }
}
if (intCustomerIndex != -1) {
    customers[intCustomerIndex].setfirstName( strFirstName);
}
\end{javacode}

% \paragraph{Main\\}

% \begin{itemize}
%     \item \javainline{String lineSeparator} is useless. 
% \end{itemize}

% * Class is meant to be a Singleton. 

% ## AllCustomers

% Very badly named class. 

% 1. We create an array of customers with size = MAX_CUSTOMERS = 100. 
%   * Do we check that number when adding? Is it possible to overflow this array?
%   * some stuff is not specified as private, protected, public (bad practice). 

% 2. We never return the Customer object, which is fair maybe we don't want to expose it? But anyway, we should be returning something about its ID. --> we have added getIntCurrentCustomerIndex
  
% ### addCustomer()

% Visibility: public
% Input: firstName, lastName, taxCode, salary
% Output: void 

% Details: 
%     * if we have not exceeded max number of customers, create a new customer passing in all the params.
%     * no checks whatsoever are done on the input params
%     * increment currentCustomerIndex
%     * if we have exceeded print error. Return.
    
% ### getCustomer(int customerID)

% Visibility: public 

% Output: a Customer() instance. Always, even if not found. If not found, we return a default Customer() which has an INVALID_ACCOUNT=-1.
% Details: 

% * i=0 is redundant, we could just define it in the for loop 
% * we're not looping correctly. We traverse in the loop n+1 elements (see `i<= intCurrentCustomerIndex`). 
% This probably does not manifest now because we either find the guy we're looking for before reaching the end, or we don't find him 
% and reach the last empty entry. 


% ### deleteCustomer(int customerID)

% Visibility: public 

% ### updateCustomerAll

% Quite surprising that this method does not call on the other update methods. This would make sure there is no code duplication, 
% plus if we later find a bug somewhere, either in updateCustomerAll() or in updateCustomerFirstName() for example, then one change 
% in code will fix both. 

% **TODO to be continued** 

% -- 


% ## Customer

% * there are no getters for field names first name, last name, tax code. 
% * not using camel case in the name of setter methods setlasname and setfirstname 
% * field parameters have a different name from the setter method. This is justifiable sometimes, but this is just a cock up. 
% * the overall desgin of the class is poor and poses some headaches for testers. Creating a customer with default constructor gives them an INVALID_ACCOUNT number. Is this in the requirements? Should we test for it? 


% =============================================================================
% CODE SNIPPETS
\pagebreak
\section{Code snippets}
\label{app:code-snippets}
% =============================================================================

% -----------------------------------------------------------------------------
\begin{figure}[H]
\centering
\begin{javacode}
/**
 * Stream to write application's output to, defaults to System.out.
 */
private final PrintStream out;

/**
 * InputStream to be used for reading user input, defaults to System.in
 */
private final InputStream in;

/**
 * default constructor
 */
public Main() {
    this(System.in, System.out);
}

/**
 * constructor taking in the InputStream and PrintStream to be used
 * @param in InputStream to read from
 * @param out PrintStream to write to
 */
public Main(InputStream in, PrintStream out) {
    this.in = in;
    this.out = out;
}
\end{javacode}
\caption{PrintStream \& InputStream in Main}
\label{code:snippet-1}
\end{figure}
% -----------------------------------------------------------------------------
\pagebreak
\section{Summary of tools used} 
\begin{tabularx}{\textwidth}{ | s | b | }
  \hline %  -----------   Header  ---------------------------------
  \tblheader{Tools} & \tblheader{Use} \\
  \hline %  ---------------------------------------------------------------------
  IntelliJ & Integrated Development Environment\\ 
  \hline %  ---------------------------------------------------------------------
  Git & Source Control\\ 
  \hline %  ---------------------------------------------------------------------
  JUnit & Unit Testing\\ 
  \hline % --------------------------------------------------------------
  Cucumber & Acceptance Testing\\ 
  \hline % --------------------------------------------------------------
  Jenkins & Automated Continuous Integration \\
  \hline %  ---------------------------------------------------------------------
  Gradle & Build and Dependency Management\\ 
  \hline % ------------------------------------------------------------
  Javadocs & Documentation \\ 
  \hline %  ---------------------------------------------------------------------
  Jacoco & Code Coverage \\ 
  \hline %  ---------------------------------------------------------------------
  \end{tabularx}



% =============================================================================
% END DOCUMENT
% =============================================================================

\end{document}

% =============================================================================
% CHECKS
% =============================================================================

% [  ] Check spelling
% [  ] Check with "grammarly"

% =============================================================================
% DRAFT
% =============================================================================