\section{Review: System Requirements}

% what type of requirements document is this, are they informal? 
It is not evident at the outset what type of requirements document these requirements belong in. If they are meant to be informal, we would kindly ask for a formal requirements specification to guide our testing, as without one, there will be muchroom for vagueness and assumptions to be made. This could lead to testing being done on a false understanding of the requirements and this in turn would be costly both in terms of money and time.  

% vocabulary not defined before the requirements. 
In formal requirements documents, key terms such as "will", "should", "must", and "may" are defined prior to the listing of the requirements in order to more precisely depict the requirement priorities.
It is good that the present requirements limit the key terms used to a subset of well-known ones (will, may, should, must) are used. This reduces the risk of vagueness emanating from (english understanding). 
Client and customer are used interchangeably in the requirements, we assume they refer to the same entity, but it would be good to put that in writing. Again "Definitions" section before the requirements would be the place to have that (IEEE).

% requirements organisation (bullets, no numbering)
The requirements are bulletized, a numbering scheme would be much more useful as that would allow for specific referencing in future discussions, documents and code. 
We will take it on ourselves to label the requirements for our own workflow, it is suggested that any additions or alterations needed on the requirements be done on the table (reference to table here). 
The first bullet in the list is not an explicit requirement, but an introductory sentence. This is misleading to the reader. 

% Some lack of coherence which lends to an overall lack of professionalism in the requirements 
Seventh bullet "The system must include a simple ‘help’ [...]" does not terminate with a full stop while the rest do. 
When outlining marital status taxing, we can see "Calculation of tax for Married people", "Calculation of tax for Single people", but the word 'divorced' is not capitalised, typo or lack of respect for divorced?  

% non-functional requirements. 
The only non-functional requirement related to the application's robustness. Should we assume there is no interest in security, performance? Surely this isn't the case for a Tax Calculation application. 

% DEFECTS and VAGUENESS in requirements 
% --------------------------------------

% 1 
[ The application must enable the storage of client’s personal details including.  ] 

There is a lack of specificity on what storage means here. Does it refer to persistent storage? In which, case a form of database on disk would be required, or is it sufficient for it to be stored in memory during application runtime? The latter would not offer persistent for if the application or machine crashed.  

If persistent storage is to be used, it would be useful for the requirements to ouline the requirements on such database, are there non-function requirements such as performance metrics that the database should respect? 

% 2 
[ The software should issue each customer a numeric identifier which must be unique to that client and never be used by a different client (that is, if a client is deleted from the system then their ID should not be used again but has to ‘die’ with them). ] 

This requirement is relatively more detailed than some of the others, given it is accompanied by an example detailing a scenario and how the application should behave with respect to that scenario. 

While 'die' is quoted, it is still an aggressive term to use in a requirement. 'Expire' is a reasonable alternative term, but we won't pedantize on this as there are more serious defects and vaguenesses in the requirements.

% 3
[ The PTMA should enable customer details to be updated. ]

This requirement can be further expanded on, especially considering that it relates to significant chunks of the code. 
What details can a user update? Should it be allowed to update a customer's account number (ID)?
Is it required that there be a separate menu command for updating each of the following: first name, last name, tax code ? What about salary? Was this a desirable or required feature?    

% 4
[ The application should be able to print lists of customer information with users being able to select blocks of customers to be printed based on their unique identifiers. ]

This requirements weakness only becomes evident when the application is tested and it becomes apparent that the Unique Identifiers generated for the created customers are never presented to the the application user for them to take note of. 

This requirement can further do with specificity to what type of block can be selected in the list to print. 
Should it be possible to use a comma-separated list of blocks to print out? This approach is not uncommon in other computer applications (e.g. specifying the list of pages to print on modern OSes is possible with "2-10,15-35"). This requirement could do with more specificity to the pages format. 

% 5
[ The application will calculate the tax amount to be paid by each client and display this
amount along with the other customer details. The amount of the tax is based on their tax
code and their current salary according to the rules below. ]



% requiremrents lacking more explanation (e.g. 3)
no expanding on how the update process should be. It seems that we want to allow the flexibility to update either of the customer's fields directly with its own command, but this does not link back to an evident requirement from the bulletised list. 

% MISSING REQUIREMENTS ? 

% showing the ID of created clients 
Requirement 4 points to the need to allow users to see customers based on their IDs, but there is no reference to how the application-generated IDs are presented to the user. This was found to be an issue during exploratory testing, and was sourced back to both a defect in the implementation and absence in the requirements. 

% no requirement to delete clients. 
There is no requirement about the user's ability to delete customers from storage. This is tacitly mentioned as part of the second requirements relating to ID uniqueness. We believe it is then an oversight on the requirements writers. 

% Adjustments and assumptions we make on the requirements. 
% Requests for improvements on the requirements. 