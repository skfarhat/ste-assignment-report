\section{Review: System Requirements}

% what type of requirements document is this, are they informal? 
It is not evident at the outset what type of requirements document these requirements belong in. If they are meant to be informal, we would kindly ask for a formal requirements specification to guide our testing, as without one, there will be much room for vagueness and assumptions to be made. This could lead to testing being done on a false understanding of the requirements and this in turn would be costly both in terms of money and time. 	
\par

% Requirements are missing "definitions" section 
% ----------------------------------------------
The requirements are missing key sections that provide the completeness required by an SRS document to properly define the requirements of the project. 	
\par

For instance, we note the absence of a definitions section which defines key terms used in the requirements. It would be useful to point out that terms "will", "should", "must", and "may" conform to a predefined standard definition (e.g. IEEE Std 610.12-1990) to highlight the priority of said requirements and avoid ambiguity in their interpretation.
More significantly though, terms more specific to the application should be explicitly defined: 

\begin{itemize} 
	\item Client and customer are used interchangeably. We assume through contextual understanding that they refer to the same entity. However, it would be good to put that in writing. Again "Definitions" section before the requirements would be the place to have that (IEEE).
	\item System and application are used interchangeably: the 2 requirements before last state "They system must [..]" while others refer to "the application". Again, we assume through contextual understanding that the terms refer to the software being tested, but request the requirements writers specify this. 
\end{itemize}
\par

% vocabulary not defined before the requirements. 
% The requirements seem to be missing a section that defines key terms used in the the different requirements such as "will", "should", "must", and "may". This is common practice in SRS documents are defined prior to the listing of the requirements in order to more precisely depict the requirement priorities.
% It is good that the present requirements limit the key terms used to a subset of well-known ones (will, may, should, must) are used. This reduces the risk of vagueness emanating from (english understanding). 
% Some SRS documents will capitalise SHOULD, MUST, MAY to leave no room for confusion. 

% -------------------------------------------------
% Requirements organisation (bullets, no numbering)
% -------------------------------------------------
\subsection{Requirements organisation}
The requirements are bulletized, a numbering scheme would be much more apt as it allows testers and reviewers of the document the ability to reference individual requirements using their identifier. In some organisation, test methods or classes are prefixed with the requirement reference they address. 

We will take it on ourselves to label the requirements for our own workflow, it is suggested that any additions or alterations needed on the requirements be done on the table (reference to table here).

% -------------------------------------------------
% Requirements doc lacks of coherence 
% which lends to an overall lack of professionalism
% --------------------------------------------------
\subsection{Requirements coherence}

The first bullet in the list is not an explicit requirement, but an introductory sentence. This is misleading to the reader. 

Seventh bullet "The system must include a simple ‘help’ [...]" does not terminate with a full stop while the rest do. 
When outlining marital status taxing, we can see "Calculation of tax for Married people", "Calculation of tax for Single people", but the word 'divorced' is not capitalised, typo or lack of respect for divorced?  

% non-functional requirements. 
The only non-functional requirement related to the application's robustness. Should we assume there is no interest in security, performance? Surely this isn't the case for a Tax Calculation application. What platforms must the system run on? Must the language to be used be Java, or can developer preference overtake in the matter? 
\par

% * Non-functional requirements have not been detailed: do we expect the system to run on any platform? are there any speed / performance requirements that need to be catered for? For example, the requirements don't mention how many clients can be persisted on the app, which may imply (unlimited), but in such a case, it would be useful to reconsider the array design, as this does not provide enough flexibility going forward. 

% * Are there no security requirements? Perhaps the ability to allow locking the app with an app is desirable? After all, the software does hold people's tax codes, rates...


% --------------------------------------
% DEFECTS and VAGUENESS in requirements 
% --------------------------------------

\subsection{Defects and Vagueness}

In this section we address imprecise requirements and what we believe to be defects in the requirements. The provided requirements have been labelled (see Appendix X) to improve readability and accessibility of the report. 

% -----------------------------------------------------------------------------
% [1]
\subsubsection{ R1 - The application must enable the storage of client’s personal details including. }

There is a lack of specificity on what storage means here. Does it refer to persistent storage? In which, case a form of database on disk would be required, or is it sufficient for it to be stored in memory during application runtime? The latter would not offer persistent for if the application or machine crashed.  

If persistent storage is to be used, it would be useful for the requirements to ouline the requirements on such database, are there non-function requirements such as performance metrics that the database should respect? 

% -----------------------------------------------------------------------------
% [2]
\subsubsection{R2 - The software should issue each customer a numeric identifier which must be unique to that client and never be used by a different client (that is, if a client is deleted from the system then their ID should not be used again but has to die with them). }

This requirement is relatively more detailed than some of the others, given it is accompanied by an example detailing a scenario and how the application should behave with respect to that scenario. 

While 'die' is quoted, it is still an aggressive term to use in a requirement. 'Expire' is a reasonable alternative term, but we won't pedantize on this as there are more serious defects and vaguenesses in the requirements.

% -----------------------------------------------------------------------------
% [3]
\subsubsection{R3 - The PTMA should enable customer details to be updated.}

This requirement can be further expanded on, especially considering that it relates to significant chunks of the code. 
What details can a user update? Should it be allowed to update a customer's account number (ID)?
Is it required that there be a separate menu command for updating each of the following: first name, last name, tax code ? What about salary? Was this a desirable or required feature?    

% requiremrents lacking more explanation (e.g. 3)
% no expanding on how the update process should be. It seems that we want to allow the flexibility to update either of the customer's fields directly with its own command, but this does not link back to an evident requirement from the bulletised list. 

% -----------------------------------------------------------------------------
% [4]
\subsubsection{R4 - The application should be able to print lists of customer information with users being able to select blocks of customers to be printed based on their unique identifiers.}

This requirements weakness only becomes evident when the application is tested and it becomes apparent that the Unique Identifiers generated for the created customers are never presented to the the application user for them to take note of. 

This requirement can further do with specificity to what type of block can be selected in the list to print. 
Should it be possible to use a comma-separated list of blocks to print out? This approach is not uncommon in other computer applications (e.g. specifying the list of pages to print on modern OSes is possible with "2-10,15-35"). This requirement could do with more specificity to the pages format. 

% -----------------------------------------------------------------------------
% [5]
\subsubsection{ The application will calculate the tax amount to be paid by each client and display this amount along with the other customer details. The amount of the tax is based on their taxcode and their current salary according to the rules below.}

This requirement is reasonably understandable. It is not however clear why it was separeted from the tax rules, seems like they could have been structured to follow each other. 

% -----------------------------------------------------------------------------
% [6]
\subsubsection{R6 -  The system must include a simple 'help' system that lists all commands}
Assuming system refers to application, there's no ambiguity in the functionality requested in this requirement. 

% -----------------------------------------------------------------------------
% [7]
\subsubsection{R7 - The system must not crash if the user enters something that they are not meant to.}
Assuming system refers to the application, the requirement is clear. This requirement relates to the application's robustness and should, in a more formal document, be placed in the non-functional requirements section along with other requirements relating to performance, robustness, security. 

% -----------------------------------------------------------------------------
% [8]
\subsubsection{R8 - TaxRules}
[ Thus, for clarity, both of the following tax codes are valid. ] 
It would be extra helpful to provide a tax code that should not be accepted (e.g. 1080MT90). In fairness the provided examples were appreciated as they helped clarify the requriement.

% The alphabetic portion of the code may consist of multiple letters drawn from the
% following set, with their corresponding meanings:
The requirement does not explicitly elicit the fact that the different letters pertaning to marital status are mutually exclusive. Similarly, it does not do this for education and number of children. 

The requirements do not specify there must be a letter from the marital status in the tax code. Intuitively, an individual must be in one of those sections hence the tax code MUST have a letter from the marital status. This is not specified in the requirement but was assumed during testing. 

The requirement does not specify the application's behaviour for the case when duplicates of the same letter are found in the tax code. We would expect the tax code parser to complain and specify this as invalid, however others may interpret "DD1000" to be apply the Divorced taxing twice. 

% common to all next sections: 

The statement: "Married people with a current salary less than £12,000 will pay a tax rate of 90\% of their base tax amount.""

alone, implies that someone earning £4,000 must pay a rate of 90\%. This is not true, in the case of this application given the other statement: "Married people with a current salary of less than £5000 will pay a tax rate of 70\% of their base tax amount.". 

We assume the intention is to say: 

* People earning between [0-5000[ should pay 70\% tax rate 
* People earning between [5000-12000] should pay 90\%

The requirements are not as well expressed here. 

% 8 - Married section 
The requirements for taxing Married people do not cover those who earn exactly £27,000. The third bullet ("less than") implies they are excluded from the 120\% tax rate, and the bullet right after ("above than") excludes them as well. 

% 8 - Single section 
Third bullet possibly mistyped. If it is not, then the third and fourth bullets are contradictory. We assume the intention was to say:  

* People earning in bracket [11k-22k[ pay a rate of 125\%
* People earning 22k and above pay rate 132\%

If third bullet is adjusted to say "less than" instead of above, then this fixes the contradiction, but still leaves this requirement not covering the case where the taxee earns £22,000 exactly (same as Married section). 

% 8 - Divorced 
Same as single section. 

The following salaries / tax codes are not covered by the requirements: 
\begin{itemize}
	\item Married and earning £27,000
	\item Single and earning £22,000 (if other typo is fixed)* (see Single Section)
	\item Divorced and earning £24,000 (if other typo is fixed) *(see Divorced Section)
	\item Single Child and earning £10,400
	\item Two children and earning £9900
	\item More than 2 children and earning £9000
	\item FT education below 24yo and earning £3000
	\item FT education 24yo and above earning £3000
\end{itemize}

% ----------------------------------------------------------------------------------------------
% MISSING REQUIREMENTS ? 

% showing the ID of created clients 
Requirement 4 points to the need to allow users to see customers based on their IDs, but there is no reference to how the application-generated IDs are presented to the user. This was found to be an issue during exploratory testing, and was sourced back to both a defect in the implementation and absence in the requirements. 

% no requirement to delete clients. 
There is no requirement about the user's ability to delete customers from storage. This is tacitly mentioned as part of the second requirements relating to ID uniqueness. We believe it is then an oversight on the requirements writers. 

% Edge case

Is it implicity that negative salaries are not a thing? How does the code deal with that? 

% Adjustments and assumptions we make on the requirements. 
\subsection{Assumptions}
\begin{itemize}
	\item We assume that earning less than zero (Paying?) person, pays no tax. i.e. min range = 0. 
\end{itemize}


% Requests for improvements on the requirements. 
% TODO: would be great if we could write the requirements as we understood them.

