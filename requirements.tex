% Requests for improvements on the requirements. 
% TODO: would be great if we could write the requirements as we understood them.
% TODO: Talk about our impression from reading the system requirements and all the vaguenesses found in there. 
% ------------------------------------------------------------------------------------------------------------
\section{System Requirements}

This section presents a review of the set system requirements, identifying those that are considered ‘unclear’, vague, or requirements that may be classified as defects in the specifications.  Once identified, assumptions will be made to manage such requirements.
\par
The source or basis of such system specifications has not been made identified during the course of this work.  Hence, should a more comprehensive reference document be made available, such as IEEE SRS, which can be reviewed, then development would be greatly simplified, and the output would present a clearer guide, with less ambiguities. Testing of specification and the basis on which these were selected are critical to the work particularly in the test phase. Test trials based on false, incomplete, or unclear specifications, may generate many false-positive or even worse false-negatives, and may prove costly both in terms of development time and development cost.

% In this section we shall review the provided requirements calling attention to mistakes and vaguenesses in the requirements, which we recognise as defects. As well as identifying them, we shall delineate any assumptions we are forced to make due to missing or imprecise requirements. 
% % Type of Req Document? Informal? 
% % -------------------------------
% It is not evident at the outset the type of specification document these requirements were extracted from; if a more complete document is available, such as an IEEE SRS, then we would kindly request access to it as it should help in filling the voids left by the current extact.  
% This step is critical before proceeding with the test process as testing based on false or incomplete understanding risks generating lots of false-positive or worse false-negatives, and would be costly both in terms of time and money. 
% \par

% -------------------------------------------------
% Requirements organisation (bullets, no numbering)
% -------------------------------------------------
\subsection{Organisation of Requirements}
Requirements are presented in the form of bullets, whereas a numbering scheme would have provided a smoother workflow, since those testing or reviewing the test document would reference individual requirements using specific identifiers.  In some organisations, test methods or test classes are prefixed with the requirement reference being addressed.
\par
A part of the work involved labelling requirements for the workflow process presented in the form of a table, as shown in Appendix \ref{app:labelled-requirements}.  Therefore as a good practice, it is recommended that additions or alterations to be made to requirements are actually amended on the reference table.

% The requirements are bulletised, a numbering scheme would be much more apt as it allows testers and reviewers of the document the ability to reference individual requirements using their identifier. In some organisation, test methods or classes are prefixed with the requirement reference they address. 
% We will take it on ourselves to label the requirements for our own workflow (see Appendix \ref{app:labelled-requirements}) and we propose that any additions or alterations to be made on the requirements be done on that table.

% TODO: Some SRS documents will capitalise SHOULD, MUST, MAY to leave no room for confusion. 

% --------------------------------------------------
% Requirements doc lacks of coherence 
% which lends to an overall lack of professionalism
% --------------------------------------------------
\subsection{Coherence of Requirements}

At a first glance, it becomes apparent that there is a lack of coherence in the structure and wording of requirements, which indicates a lack of a formal and systematic approach in the software engineering process.  A positive spin-off of this surprisingly may be that it alerts the Tester on what he is to expect in the quality of software under test, hence may question project management procedures in place, and as a result the Developer’s approach and end-result.
\par
Below some of the non-coherences observed are listed. These are by no means critical, but may be flagged as defects.
\begin{itemize}
    \item The \nth{1} bullet in the list (\textit{"The PTMA must implement a Personal Tax Management System to meet the following requirements."}) is not an explicit requirement, but an introductory sentence which would better be presented without a bullet and is misleading to the reader. 
    \item The \nth{7} bullet in the list (\textit{"The system must include a simple ‘help’ system that lists all commands"}) does not terminate with a full stop while the rest do. 
    \item Requirements (\REightFive to \REightTen) all commence with \textit{Calculation of tax for people}, whereas the last two requirements (\REightEleven to \REightTwelve) commence with \textit{Calculation for people}. It is not evident why different wording was opted for between these requirements. 
    \item The bullet detailing the taxing of divorced people (\REightSeven) does not capitalise the word "divorced", but does capitalise "Single" and "Married". Readers should attempt their not to take this as a lack of respect for divorced individuals. 
%   % in the When outlining marital status taxing, we can see "Calculation of tax for Married people", "Calculation of tax for Single people", but the word 'divorced' is not capitalised, typo or lack of respect for divorced?  
\end{itemize}

% OLD
% At first read, a lack of coherence in the structure and wording of the requirements is clear, depicting an overall lack of formality and professionalism in the software engineering process. 
% A positive consequence of this, perhaps surprisingly, is that it puts the tester on the alert for what to expect of the quality of the software to be tested, making them question the meticulousness of the project managers and in turn the developers.
% \par
% Below we outline some of the observed incoherences. They are not catastrophic by any means and need not be raised as defects, but are still worth raising:
% \begin{itemize}
%     \item The \nth{1} bullet in the list (\textit{"The PTMA must implement a Personal Tax Management System to meet the following requirements."}) is not an explicit requirement, but an introductory sentence which should not have been bulletised. This is misleading to the reader seein as most other bullets are requirements. 
% 	\item The \nth{7} bullet in the list (\textit{"The system must include a simple ‘help’ system that lists all commands"}) does not terminate with a full stop while the rest do. 
% 	\item Requirements (\REightFive to \REightTen) all begin with \textit{Calculation of tax for people}, whereas the last two requirements (\REightEleven to \REightTwelve) begin with \textit{Calculation for people}. It is not clear why there is a distinction in the wording of these requirements. 
% 	\item The bullet detailing the taxing of divorced people (\REightSeven) does not capitalise the word "divorced", but does capitalise "Single" and "Married". The reader will attempt their best not to take this as a lack of respect for divorced individuals. 
% 	% in the When outlining marital status taxing, we can see "Calculation of tax for Married people", "Calculation of tax for Single people", but the word 'divorced' is not capitalised, typo or lack of respect for divorced?  
% \end{itemize}

% ----------------------------------------------
% Non-Functional Requirements
% ----------------------------------------------
\subsection{Non-functional requirements}
\label{sec:non-functional-requirements}
It is understood that performance testing, security, and other non-functional requirements are not within the scope of testing for this project;
However, including these in the system requirements review is essential as these constitute an integral part of the requirements development process. It is usually expected that requirements on security, performance, compatibility and robustness are elicited clearly in a separate non-functional requirements section.  However, it may well be that the scope of testing is restricted to functional aspects.
\par
Despite this, the following is identified (\RSeven:\textit{"The system must not crash if the user enters something that they are not meant to."}) as a non-functional requirement pertaining to the robustness of the application. In a more formal document, this requirement would be separated from functional requirements.

%
% Ignorable: 
% ----------
% What platforms must the system run on? Must the language to be used be Java, or can developer preference overtake in the matter? 
% * Non-functional requirements have not been detailed: do we expect the system to run on any platform? are there any speed / performance requirements that need to be catered for? For example, the requirements don't mention how many clients can be persisted on the app, which may imply (unlimited), but in such a case, it would be useful to reconsider the array design, as this does not provide enough flexibility going forward. 
% * Are there no security requirements? Perhaps the ability to allow locking the app with an app is desirable? After all, the software does hold people's tax codes, rates...
\par

% ----------------------------------------------
% Key term definitions
% ----------------------------------------------
\subsection{Key term definitions}
\label{sec:key-term-definitions}

Some key sections required in a typical SRS document are missing from the requirements.  As an example, a "Definitions" section outlining the meaning of key terms used in the requirements is missing.  It is also important to note that terms such as "will", "should", "must", or "may", must conform to a predefined standard definition (e.g. IEEE Std), in order to ascertain their priority over each other, hence avoiding confusion in the interpretation of each term. The importance of this is not to be under-estimated particularly that it is likely that the audience of the document will include non-native English speakers.
\par
Furthermore, the exact explicit definition of the below keywords is absolutely essential, in view of the fact that terms appear to be used interchangeably:
\begin{itemize}
    \item Client, Customer, and Person - Client is used in requirements (\ROne, \RTwo, \RFive), Customer is used in (\RTwo, \RThree, \RFour, \RFive), and Person is used in all (\REight) requirements. \RFive \space interchangeably uses Customer and Client within the same sentence. 
    \item System and Application - System is used in (\RTwo, \RSix, \RSeven), while Application is used in (\ROne, \RFour, \RFive). 
\end{itemize}
It is assumed through contextual understanding that these refer to the same entity.  However, confirmation and a more precise definition of the terms must presented in a formal document.
\par

% It is good that the present requirements limit the key terms used to a subset of well-known ones (will, may, should, must) are used. This reduces the risk of vagueness emanating from (english understanding). 

% --------------------------------------
% DEFECTS and VAGUENESS in requirements 
% --------------------------------------
\subsection{Defects and Vaguenesses}

In this section we address imprecise or 'loose' requirements, which are believed to truly form defects in requirements.
These have been labelled (see Appendix \ref{app:labelled-requirements}) to improve readability and accessibility of the report.

% [1] -----------------------------------------------------------------------------
\subsubsection{The application must enable the storage of client’s personal details including [R1]}
The term storage is not clear in that it does not specify where details are to be stored. 
Storage may be permanent (e.g. database, file on system, etc.) or in-memory surviving only until the next launch or boot up. 
\par
If persistent storage is to be used, then some more details on the how this storage should behave, how robust, portable, responsive it should be would be needed. 
\par
For testing purposes within this work, it is assumed that the storage of details need only be in-memory when the application is running, as such, no testing has been done on permanent-storage.

% [2] -----------------------------------------------------------------------------
\subsubsection{The software should issue each customer a numeric identifier [R2]}
This requirement was found to be clear and unambiguous. 
Minor comment: the usage of the word 'die' did not seem entirely appropriate, even if quoted; however, software does tend to use questionable wordage (think of the POSIX function kill), so this is a non-issue.  

% [3] -----------------------------------------------------------------------------
\subsubsection{The PTMA should enable customer details to be updated [R3]}
This requirement was found to be clear and unambiguous at first read. 
However, after noting that the implementation contains sizeable chunks of code pertaining to client updates (four menu commands each with its own method definiton), testers would question whether there is a lack of specificity from the requirements perspective, or going over the requirements by the developers. 
\par
In both cases, it is not clear and surprising why for example the 'salary' field cannot be updated with one command, as the case for 'first-name', 'last-name' and 'tax-code'.  
\par
Moreover, this requirement may be improved by specifying that customer IDs should not be possible to update in order to avoid any possible confusion during implementation. 

% [4] -----------------------------------------------------------------------------
\subsubsection{The application should be able to print lists of customer information [R4]}
% with users being able to select blocks of customers to be printed based on their unique identifiers.}
This requirement is clear and unambiguous. 
By being less specific on how the user is expected to specify the blocks to print, the requirement provides developers the flexibility, which is needed seeing as (\RFour) has strong dependency on the choice of identifiers used in (\RTwo). If non-integer identifiers were used in \RTwo \space then printing blocks of users may have to be done using comma-separated input from the user. 

% [5] -----------------------------------------------------------------------------
\subsubsection{The application will calculate the tax amount to be paid by each client [R5]} 
% and display this amount along with the other customer details. The amount of the tax is based on their taxcode and their current salary according to the rules below.}
This requirement is clear and unambiguous. 
However, it remains unclear why it was separated from (R8.x) the tax rules, seems like they could have been structured to follow each other. 

% [6] -----------------------------------------------------------------------------
\subsubsection{The system must include a simple 'help' system that lists all commands [R6]}

This requirement was found to be clear and unambiguous, assuming 'system' and 'application' refer to the same thing (see Section \ref{sec:key-term-definitions}). 

% [7] -----------------------------------------------------------------------------
\subsubsection{The system must not crash if the user enters something that they are not meant to [R7]}
Assuming 'system' refers to the application, the requirement is clear. 
This requirement relates to the application's robustness and should, in a more formal document  be placed in the non-functional requirements section along with other requirements relating to performance, robustness, security (see section \ref{sec:non-functional-requirements}).

% [8] -----------------------------------------------------------------------------
\subsubsection{R8 - Tax Rules}

In this sub-section we will comment on defects found in requirements (R8.x) without necessarily delving into each requirement individually since many of them share the same defects. 
\par
% ----------------------------------
% Mutual exclusivity of tax brackets 
% ----------------------------------
\textbf{Mutual exclusivity of tax brackets \\}
\\
Requirement (\REightFour) does not cite the existence of any mutual exclusivity between tax letters in the same tax groups i.e. between: 
\begin{itemize}[noitemsep]
	\item 'M', 'S', and 'D' 
	\item 'C', 'E', and 'F'
	\item 'T' and 'U'
\end{itemize}
This will be assumed considering a person cannot have both "2 children" and "1 child" at the same time, and so cannot be in both groups. Similarly, one's martial status cannot be both single and divorced, hence the mutual exclusivity.  

% --------------------------------------
% Marital Status is required in tax code
% --------------------------------------
\textbf{Marital status is required in tax code \\}
\\
Requirement (\REightFour) does not specify that the tax code MUST contain a martial status. There are many reasons why this can be justifiably assumed to be a missing requirement: 
\begin{enumerate}
	\item Requirements (\REightEight) and onwards all indicate that the tax to be calculated for their respective brackets has to be done after the Marital status tax has been applied. This suggests that some tax due to Marital status must have been applied before. 
	\item If it were possible for the Martial status to be absent from the tax code, then it would become possible for a tax code to not have any letters at all. This would conflict with requirement (\REightTwo). % This can happen if they have no Marital status, no children and are not in education. 
	% \item The reader is not aware of a marital status that does not belong outside of married, divroced, single. Common sense employed here. 
\end{enumerate}

% -----------------
% Unusual tax rates
% -----------------
\textbf{Unusual tax rates \& negative salaries \\}
\\
The unusual tax rates were brought to the attention of the reader in the assignment notes, hence there is no call for pointing them out as a defect. However, in light of the possiblity of a tax payer earning a salary and paying all of it and more to the government (when tax rate > 100\%), one must ask whether negative salaries as a concept exist in Failovia. 
\par
It is not clear from the requirements how the application should respond to a user-provided negative salary or base amount. As testers, we would expect two valid application responses to negative salaries: 

\begin{enumerate}
	\item Calculate the tax from the negative salary leading to a negative tax. In practical terms, this means the tax authorities owe the client some money.
	\item Provide a user error when a negative salary of negative tax amount is input. 
\end{enumerate}

Which approach the application should undertake should be laid out in the requirements.

% --------------------------
% Contradictory requirements
% --------------------------
\textbf{Contradictory requirements\\}
\\
The below requirement (\REightFive) does not correctly  is incorrectly worded and would could have been interpreted  can justifiably be interpreted differently than it has been by the developers who we believe overlooked the issues inherent in the requirement. 
\par
%TODO: format this differently from the rest of the text - but make it look good. 
Calculation of tax for Married people: one of the following four bands will apply:
\begin{itemize}[noitemsep]
	\item Married people with a current salary of less than £5000 will pay a tax rate of
	70\% of their base tax amount.
	\item Married people with a current salary less than £12,000 will pay a tax rate of
	90\% of their base tax amount.
	\item Married people with a current salary less than £27,000 will pay a tax rate of
	120\% of their base tax amount
	\item Married people with a current salary above than £27,000 will pay a tax rate
	of 130\% of their base tax amount
\end{itemize}	

We note from the above that each bullet after the first one contradicts the bullets before it: for someone earning £3,000 in salary, the first bullet implies they should pay a tax rate of 70\%, the second overrides the first bullet and implies they should pay 90\%, the third one 120\% and the fourth one 130\%. It is believed that the intention behind the bullets is to establish tax brackets as below: 
\begin{itemize}[noitemsep]
	\item People earning between £0 - £5,000 should pay 70\% tax rate 
	\item People earning between £5,000 - £12,000 should pay 90\%
	\item People earning between £12,000 - £27,000 should pay 120\%
	\item People earning over £27,000 should pay 130\%
\end{itemize}
The same applies for all subsequent requirements \REightSix \space to \REightTwelve. 

% Different tax rates for the same bracket
Requirements (\REightSix \space and \REightSeven) have contradictory \nth{3} and \nth{4} bullets which state different tax rates for the same tax brackets. Looking at the rest of the requirements we infer that this defect is due to a typo in the \nth{3} bullet of \REightSix \space and \REightSeven. 

We assume the intention is below: 
\begin{itemize}[noitemsep]
	\item People earning in bracket £11,000 - £22,000 pay a rate of 125\%.
	\item People earning £22,000 and above pay rate 132\%.
\end{itemize}

% -------------
% EDGE SALARIES 
% -------------
\textbf{Edge salaries\\}
\\
% 8 - Married section 
The requirements for taxing Married people do not cover those who earn exactly £27,000. The third bullet ("less than") implies they are excluded from the 120\% tax rate, and the bullet right after ("above than") excludes them as well. 
\par
We summarise below salaries that are not covered by the requirements:  
\begin{itemize}%[noitemsep]
	\item Married and earning £27,000
	\item Single and earning £22,000
	\item Divorced and earning £24,000
	\item Single Child and earning £10,400
	\item Two children and earning £9,900
	\item More than 2 children and earning £9,000
	\item FT education below 24 and earning £3,000
	\item FT education 24 and above earning £3,000
\end{itemize}

% Would be helpful 
% [ Thus, for clarity, both of the following tax codes are valid. ] 
% It would be extra helpful to provide a tax code that should not be accepted (e.g. 1080MT90). In fairness the provided examples were appreciated as they helped clarify the requriement.

% --------------------------------------
% MISSING REQUIREMENTS ? 
% --------------------------------------
\subsection{Missing requirements}

% Delete clients. 
There are no requirements for the user's ability to delete customers from storage. This is tacitly mentioned as part of the second requirements relating to ID uniqueness. It is believed that it is then an oversight on the requirements writers. 
\par
% Show ID of clients. 
Requirement (\RFour) points to the need for allowing users to see customers based on their IDs. However, there is no reference to how the application-generated IDs are presented to the user. This was found to be an issue during exploratory testing, and was sourced back to both a defect in the implementation and absence in the requirements. 
\par
% Edge case: Negative salaries. 
Is it implicity that negative salaries are not a thing? How could the code deal with that? 
\par
% Edge case: taxcode with duplicate tax letters
The requirement does not specify the application's behaviour for the case when duplicates of the same letter are found in the tax code. We would expect the tax code parser to complain and specify this as invalid, however others may interpret "DD1000" to be apply the Divorced taxing twice. 

% --------------------------------------
% Assumptions
% --------------------------------------

\subsection{Assumptions}
Requirements were reviewed and assumptions made in the testing are listed as in the following.
\begin{itemize}
    \item It is assumed that requirements do not require persistent storage of customer details.
    \item Tax letters in the same group are mutually exclusive. 
    \item Tax code MUST contain a Marital status letter. 
    \item Tax for Married people - It is assumed "above than or equal to £27,000" pay 130\%. 
    \item Tax for Single people - It is assumed that "less than £22,000" pay 125\%. 
    \item Tax for Single people - It is assumed that "above than or equal £22,000" pay 132\%. 
    \item Tax for Divorced people - It is assumed that "less than £24,000" pay 95\%.
    \item Tax for Divorced people - It is assumed that "above than or equal £24,000" pay 110\%.
    \item It is assumed that that negative salaries and base amounts are invalid.
    \item It is assumed that lower case tax-letters are NOT equivalent to their respective upper-case letter. 
\end{itemize}

% --------------------------------------
% END OF REQUIREMENTS
% --------------------------------------