% =============================================================================
\pagebreak
\section{Code critique}
\label{app:code-critique}

This section provides a critique of the code implementation found in the application. Issues pointed here do not reference functional defects of the code, but rather flaws in code style, design and good coding practice. 

\subsection{Documentation}
There is a very clear lack of documentation across the source code. Methods do not have javadoc, most method implementations do no supply them either excepting for a few inline comments. 

\subsection{Code Style}
There's no code style consistency: 
\begin{itemize}
    \item some methods put spaces between open bracket and first param and closing bracket and last parameter. Other methods do not obey this. 
    \item some methods use canonical camel case while others break this (newcustomer vs. addCustomer)
    \item Implementation oddities, as if the developer is not familiar with Java programming: 
        * `String name = new String()` is not often seen as it is sufficient to use `String name = "";' which is equivalent
        * int i = 0; then using for(i=0; i < n; i++). As i is not used outside of the loop, it does not need to be defined outside. 
    \item AllCustomers is a badly named class, CustomerManagement is a better alternative. 
\end{itemize}
% -------------------
\paragraph{AccountNumbers}
Is not a well implemented Singleton pattern because the constructor is made protected and not private, meaning other classes in the same package can still instantiate it.
% -------------------
\paragraph{AllCustomers}
\begin{itemize}
    \item method getCustomer() returns a Customer object that has an "INVALID\_ID" when no such customer is present. This is unusual in that most common implementations would either throw an exception or return null. This implementation would be fine and accepted were it to be more documented. 

    \item method deleteCustomer(). why not just place `customers[i] = customers[i+1];` in the first if? 

    \item updateCustomerXXXXX: all of these methods start off finding the customer, we should create a private method that returns the customer for a given ID. Also, discussing the current implementation of finding that customer; the below: 
\end{itemize}

\begin{javacode}
int i = 0;
int intCustomerIndex = 0;
boolean blFound = false;
for (i=0; i< intCurrentCustomerIndex; i++) {
    if (customers[i].getAccountNum() == intCustomerID) {
        blFound = true;
        intCustomerIndex = i;
        break;
    }
}
if (blFound == true) {
    customers[intCustomerIndex].setfirstName( strFirstName);
}
\end{javacode}
can be written as below: 
\begin{javacode}
int intCustomerIndex = -1;
for (int i=0; i< intCurrentCustomerIndex; i++) {
    if (customers[i].getAccountNum() == intCustomerID) {
        intCustomerIndex = i;
        break;
    }
}
if (intCustomerIndex != -1) {
    customers[intCustomerIndex].setfirstName( strFirstName);
}
\end{javacode}

\paragraph{Main}
\begin{itemize}
    \item A new instance of AllCustomers is created every time IntepretCommand is run(), which happens everytime some form of exception is thrown, because that's the only way to break out of the `while(true)` loop in InterpretCommand. 
    \item String lineSeparator is useless. Totally useless.
\end{itemize}

% developers should learn about non-capturing groups in regex.

% * string handling can be improved by using stringbuffer (improvement)
% * Singleton pattern is useless and doesn't add anything in the system. 
% * Class is meant to be a Singleton. 
% * We start account numbers at 1.
% * We can have `newAccountNum()` called without an instance. Then why do we need the AccountNumbers class in the first place?

% ## AllCustomers

% Very badly named class. 

% 1. We create an array of customers with size = MAX_CUSTOMERS = 100. 
%   * Do we check that number when adding? Is it possible to overflow this array?
%   * some stuff is not specified as private, protected, public (bad practice). 

% 2. We never return the Customer object, which is fair maybe we don't want to expose it? But anyway, we should be returning something about its ID. --> we have added getIntCurrentCustomerIndex
  
% ### addCustomer()

% Visibility: public
% Input: firstName, lastName, taxCode, salary
% Output: void 

% Details: 
%     * if we have not exceeded max number of customers, create a new customer passing in all the params.
%     * no checks whatsoever are done on the input params
%     * increment currentCustomerIndex
%     * if we have exceeded print error. Return.
    
% ### getCustomer(int customerID)

% Visibility: public 

% Output: a Customer() instance. Always, even if not found. If not found, we return a default Customer() which has an INVALID_ACCOUNT=-1.
% Details: 

% * i=0 is redundant, we could just define it in the for loop 
% * we're not looping correctly. We traverse in the loop n+1 elements (see `i<= intCurrentCustomerIndex`). 
% This probably does not manifest now because we either find the guy we're looking for before reaching the end, or we don't find him 
% and reach the last empty entry. 


% ### deleteCustomer(int customerID)

% Visibility: public 

% ### updateCustomerAll

% Quite surprising that this method does not call on the other update methods. This would make sure there is no code duplication, 
% plus if we later find a bug somewhere, either in updateCustomerAll() or in updateCustomerFirstName() for example, then one change 
% in code will fix both. 

% **TODO to be continued** 

% -- 


% ## Customer

% * there are no getters for field names first name, last name, tax code. 
% * not using camel case in the name of setter methods setlasname and setfirstname 
% * field parameters have a different name from the setter method. This is justifiable sometimes, but this is just a cock up. 
% * the overall desgin of the class is poor and poses some headaches for testers. Creating a customer with default constructor gives them an INVALID_ACCOUNT number. Is this in the requirements? Should we test for it? 
