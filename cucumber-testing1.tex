\subsubsection{File structure}

Briefly, cucumber testing works by first defining a feature under test in a feature file. A feature may have several scenarios with each scenario defining the given input, the steps defining a state transition, and the expected outcome from these transitions. Keywords: Given, And, Then, When are used in the feature file. 
\par
To translate those into actual code, Java methods with annotations (@Given, @And, @Then, @When) and a regular expression the step definition are created. Each method implements translates the step definition into the equivalent executing code and makes the assertions and checks it needs to. 
\par
\textbf{File structure}
\begin{itemize}
    \item \javainline{src/test/java/testing/accpetance/*.java}: java test implementation for step definitions 
    \item \javainline{src/test/resources/*.feature}: feature files defining features under test. 
\end{itemize}

Each of the feature files in resources has their own FeatureXStepDefs which defines their methods. Except for Feature7. As Feature7 is not actually a feature but more of a non-functional requirement, it consists of scenarios which span all previous features. 

\subsubsection{Splitting the code into many classes - and CommonStepDefs}
As the number of feature scenarios grew, so did the number of methods implementing each step in those scenarios. 
We moved from having a single StepDefs.java class that defined all steps to separate FeatureXStepDefs classes each of which defined the steps particular to a single feature that is under test.
These FeatureXStepDefs classes shared lots of code in common, with regards to the helper methods they would need to call and the member variables they relied on to validate the app's state. 
\par
In light of this, a single java class (CommonStepDefs) housing all common code needed to be created and linked to the other FeatureXStepDefs classes. The class contains the following: 

\begin{itemize}
    \item \javainline{applicationHasStarted}
    \item \javainline{userEnters}
    \item Many other helper methods that are used by each Feature class such as 
    \begin{itemize}
        \item \javainline{resetOutStream}, 
        \item \javainline{closeApp}, 
        \item \javainline{waitForMainToBeReady}.
    \end{itemize}
\end{itemize}


\par
\textbf{Linking Common class with Feature classes\\}
Inheritance was considered as a means of linking Common to Feature classes, however, cucumber does not support classes defining stepdefs to be inherited from as this creates multiple instances of the stepdef method making it impossible for Cucumber to choose which one to call during scenario execution. 
\par 
Next, we considered composing the Common class in each Feature class, and this was made possible through a cucumber dependency titled PicoContainer. With this library, cucumber is able to create a CommonStepDefs object for every Feature class and injecting it into the Feature constructor so long as the constructor was implemented to take in a CommonStepDefs object. Figure \ref{code:cucumber-piccontainer-di} is an extract of all that is required for this to be achieved.

\begin{figure}[H]
\begin{javacode}
/** injected through constructor by Cucumber PicoContainer */
private CommonStepDefs common;

/** constructor */ 
public Feature1StepDefs(CommonStepDefs common) {
    this.common = common;
}
\end{javacode}
\caption{Dependency injection - CommonStepDefs in Feature1StepDefs}    
\label{code:cucumber-piccontainer-di}
\end{figure}
% ---------------------------------------------------------------------------
 
% Cucumber test suite structure 
% The structure of the code in the Cucumber suite. 
% Refactoring of our Cucumber test suite. 
% there's a better way to test when adding customer and that is to get the Customer object back rather on relyin on the output of the command line to get the object back 